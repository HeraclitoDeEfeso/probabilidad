\documentclass[11pt]{article}

%%%%%%%%%% TODO TOMADO DEL PREAMBULO DE NBCONVERT %%%%%%%%%%

    \usepackage[breakable]{tcolorbox}
    \usepackage{parskip} % Stop auto-indenting (to mimic markdown behaviour)
    
    \usepackage{iftex}
    \ifPDFTeX
    	\usepackage[T1]{fontenc}
    	\usepackage{mathpazo}
    \else
    	\usepackage{fontspec}
    \fi

    % Basic figure setup, for now with no caption control since it's done
    % automatically by Pandoc (which extracts ![](path) syntax from Markdown).
    \usepackage{graphicx}
    % Maintain compatibility with old templates. Remove in nbconvert 6.0
    \let\Oldincludegraphics\includegraphics
    % Ensure that by default, figures have no caption (until we provide a
    % proper Figure object with a Caption API and a way to capture that
    % in the conversion process - todo).
    \usepackage{caption}
    \DeclareCaptionFormat{nocaption}{}
    \captionsetup{format=nocaption,aboveskip=0pt,belowskip=0pt}

    \usepackage{float}
    \floatplacement{figure}{H} % forces figures to be placed at the correct location
    \usepackage{xcolor} % Allow colors to be defined
    \usepackage{enumerate} % Needed for markdown enumerations to work
    \usepackage{geometry} % Used to adjust the document margins
    \usepackage{amsmath} % Equations
    \usepackage{amssymb} % Equations
    \usepackage{textcomp} % defines textquotesingle
    % Hack from http://tex.stackexchange.com/a/47451/13684:
    \AtBeginDocument{%
        \def\PYZsq{\textquotesingle}% Upright quotes in Pygmentized code
    }
    \usepackage{upquote} % Upright quotes for verbatim code
    \usepackage{eurosym} % defines \euro
    \usepackage[mathletters]{ucs} % Extended unicode (utf-8) support
    \usepackage{fancyvrb} % verbatim replacement that allows latex
    \usepackage{grffile} % extends the file name processing of package graphics 
                         % to support a larger range
    \makeatletter % fix for old versions of grffile with XeLaTeX
    \@ifpackagelater{grffile}{2019/11/01}
    {
      % Do nothing on new versions
    }
    {
      \def\Gread@@xetex#1{%
        \IfFileExists{"\Gin@base".bb}%
        {\Gread@eps{\Gin@base.bb}}%
        {\Gread@@xetex@aux#1}%
      }
    }
    \makeatother
    \usepackage[Export]{adjustbox} % Used to constrain images to a maximum size
    \adjustboxset{max size={0.9\linewidth}{0.9\paperheight}}

    % The hyperref package gives us a pdf with properly built
    % internal navigation ('pdf bookmarks' for the table of contents,
    % internal cross-reference links, web links for URLs, etc.)
    \usepackage{hyperref}
    % The default LaTeX title has an obnoxious amount of whitespace. By default,
    % titling removes some of it. It also provides customization options.
    \usepackage{titling}
    \usepackage{longtable} % longtable support required by pandoc >1.10
    \usepackage{booktabs}  % table support for pandoc > 1.12.2
    \usepackage[inline]{enumitem} % IRkernel/repr support (it uses the enumerate* environment)
    \usepackage[normalem]{ulem} % ulem is needed to support strikethroughs (\sout)
                                % normalem makes italics be italics, not underlines
    \usepackage{mathrsfs}
    

    
    % Colors for the hyperref package
    \definecolor{urlcolor}{rgb}{0,.145,.698}
    \definecolor{linkcolor}{rgb}{.71,0.21,0.01}
    \definecolor{citecolor}{rgb}{.12,.54,.11}

    % ANSI colors
    \definecolor{ansi-black}{HTML}{3E424D}
    \definecolor{ansi-black-intense}{HTML}{282C36}
    \definecolor{ansi-red}{HTML}{E75C58}
    \definecolor{ansi-red-intense}{HTML}{B22B31}
    \definecolor{ansi-green}{HTML}{00A250}
    \definecolor{ansi-green-intense}{HTML}{007427}
    \definecolor{ansi-yellow}{HTML}{DDB62B}
    \definecolor{ansi-yellow-intense}{HTML}{B27D12}
    \definecolor{ansi-blue}{HTML}{208FFB}
    \definecolor{ansi-blue-intense}{HTML}{0065CA}
    \definecolor{ansi-magenta}{HTML}{D160C4}
    \definecolor{ansi-magenta-intense}{HTML}{A03196}
    \definecolor{ansi-cyan}{HTML}{60C6C8}
    \definecolor{ansi-cyan-intense}{HTML}{258F8F}
    \definecolor{ansi-white}{HTML}{C5C1B4}
    \definecolor{ansi-white-intense}{HTML}{A1A6B2}
    \definecolor{ansi-default-inverse-fg}{HTML}{FFFFFF}
    \definecolor{ansi-default-inverse-bg}{HTML}{000000}

    % common color for the border for error outputs.
    \definecolor{outerrorbackground}{HTML}{FFDFDF}

    % commands and environments needed by pandoc snippets
    % extracted from the output of `pandoc -s`
    \providecommand{\tightlist}{%
      \setlength{\itemsep}{0pt}\setlength{\parskip}{0pt}}
    \DefineVerbatimEnvironment{Highlighting}{Verbatim}{commandchars=\\\{\}}
    % Add ',fontsize=\small' for more characters per line
    \newenvironment{Shaded}{}{}
    \newcommand{\KeywordTok}[1]{\textcolor[rgb]{0.00,0.44,0.13}{\textbf{{#1}}}}
    \newcommand{\DataTypeTok}[1]{\textcolor[rgb]{0.56,0.13,0.00}{{#1}}}
    \newcommand{\DecValTok}[1]{\textcolor[rgb]{0.25,0.63,0.44}{{#1}}}
    \newcommand{\BaseNTok}[1]{\textcolor[rgb]{0.25,0.63,0.44}{{#1}}}
    \newcommand{\FloatTok}[1]{\textcolor[rgb]{0.25,0.63,0.44}{{#1}}}
    \newcommand{\CharTok}[1]{\textcolor[rgb]{0.25,0.44,0.63}{{#1}}}
    \newcommand{\StringTok}[1]{\textcolor[rgb]{0.25,0.44,0.63}{{#1}}}
    \newcommand{\CommentTok}[1]{\textcolor[rgb]{0.38,0.63,0.69}{\textit{{#1}}}}
    \newcommand{\OtherTok}[1]{\textcolor[rgb]{0.00,0.44,0.13}{{#1}}}
    \newcommand{\AlertTok}[1]{\textcolor[rgb]{1.00,0.00,0.00}{\textbf{{#1}}}}
    \newcommand{\FunctionTok}[1]{\textcolor[rgb]{0.02,0.16,0.49}{{#1}}}
    \newcommand{\RegionMarkerTok}[1]{{#1}}
    \newcommand{\ErrorTok}[1]{\textcolor[rgb]{1.00,0.00,0.00}{\textbf{{#1}}}}
    \newcommand{\NormalTok}[1]{{#1}}
    
    % Additional commands for more recent versions of Pandoc
    \newcommand{\ConstantTok}[1]{\textcolor[rgb]{0.53,0.00,0.00}{{#1}}}
    \newcommand{\SpecialCharTok}[1]{\textcolor[rgb]{0.25,0.44,0.63}{{#1}}}
    \newcommand{\VerbatimStringTok}[1]{\textcolor[rgb]{0.25,0.44,0.63}{{#1}}}
    \newcommand{\SpecialStringTok}[1]{\textcolor[rgb]{0.73,0.40,0.53}{{#1}}}
    \newcommand{\ImportTok}[1]{{#1}}
    \newcommand{\DocumentationTok}[1]{\textcolor[rgb]{0.73,0.13,0.13}{\textit{{#1}}}}
    \newcommand{\AnnotationTok}[1]{\textcolor[rgb]{0.38,0.63,0.69}{\textbf{\textit{{#1}}}}}
    \newcommand{\CommentVarTok}[1]{\textcolor[rgb]{0.38,0.63,0.69}{\textbf{\textit{{#1}}}}}
    \newcommand{\VariableTok}[1]{\textcolor[rgb]{0.10,0.09,0.49}{{#1}}}
    \newcommand{\ControlFlowTok}[1]{\textcolor[rgb]{0.00,0.44,0.13}{\textbf{{#1}}}}
    \newcommand{\OperatorTok}[1]{\textcolor[rgb]{0.40,0.40,0.40}{{#1}}}
    \newcommand{\BuiltInTok}[1]{{#1}}
    \newcommand{\ExtensionTok}[1]{{#1}}
    \newcommand{\PreprocessorTok}[1]{\textcolor[rgb]{0.74,0.48,0.00}{{#1}}}
    \newcommand{\AttributeTok}[1]{\textcolor[rgb]{0.49,0.56,0.16}{{#1}}}
    \newcommand{\InformationTok}[1]{\textcolor[rgb]{0.38,0.63,0.69}{\textbf{\textit{{#1}}}}}
    \newcommand{\WarningTok}[1]{\textcolor[rgb]{0.38,0.63,0.69}{\textbf{\textit{{#1}}}}}
    
    
    % Define a nice break command that doesn't care if a line doesn't already
    % exist.
    \def\br{\hspace*{\fill} \\* }
    % Math Jax compatibility definitions
    \def\gt{>}
    \def\lt{<}
    \let\Oldtex\TeX
    \let\Oldlatex\LaTeX
    \renewcommand{\TeX}{\textrm{\Oldtex}}
    \renewcommand{\LaTeX}{\textrm{\Oldlatex}}
    % Document parameters
    % Document title
    \title{parte2}
    
    
    
    
    
% Pygments definitions
\makeatletter
\def\PY@reset{\let\PY@it=\relax \let\PY@bf=\relax%
    \let\PY@ul=\relax \let\PY@tc=\relax%
    \let\PY@bc=\relax \let\PY@ff=\relax}
\def\PY@tok#1{\csname PY@tok@#1\endcsname}
\def\PY@toks#1+{\ifx\relax#1\empty\else%
    \PY@tok{#1}\expandafter\PY@toks\fi}
\def\PY@do#1{\PY@bc{\PY@tc{\PY@ul{%
    \PY@it{\PY@bf{\PY@ff{#1}}}}}}}
\def\PY#1#2{\PY@reset\PY@toks#1+\relax+\PY@do{#2}}

\expandafter\def\csname PY@tok@w\endcsname{\def\PY@tc##1{\textcolor[rgb]{0.73,0.73,0.73}{##1}}}
\expandafter\def\csname PY@tok@c\endcsname{\let\PY@it=\textit\def\PY@tc##1{\textcolor[rgb]{0.25,0.50,0.50}{##1}}}
\expandafter\def\csname PY@tok@cp\endcsname{\def\PY@tc##1{\textcolor[rgb]{0.74,0.48,0.00}{##1}}}
\expandafter\def\csname PY@tok@k\endcsname{\let\PY@bf=\textbf\def\PY@tc##1{\textcolor[rgb]{0.00,0.50,0.00}{##1}}}
\expandafter\def\csname PY@tok@kp\endcsname{\def\PY@tc##1{\textcolor[rgb]{0.00,0.50,0.00}{##1}}}
\expandafter\def\csname PY@tok@kt\endcsname{\def\PY@tc##1{\textcolor[rgb]{0.69,0.00,0.25}{##1}}}
\expandafter\def\csname PY@tok@o\endcsname{\def\PY@tc##1{\textcolor[rgb]{0.40,0.40,0.40}{##1}}}
\expandafter\def\csname PY@tok@ow\endcsname{\let\PY@bf=\textbf\def\PY@tc##1{\textcolor[rgb]{0.67,0.13,1.00}{##1}}}
\expandafter\def\csname PY@tok@nb\endcsname{\def\PY@tc##1{\textcolor[rgb]{0.00,0.50,0.00}{##1}}}
\expandafter\def\csname PY@tok@nf\endcsname{\def\PY@tc##1{\textcolor[rgb]{0.00,0.00,1.00}{##1}}}
\expandafter\def\csname PY@tok@nc\endcsname{\let\PY@bf=\textbf\def\PY@tc##1{\textcolor[rgb]{0.00,0.00,1.00}{##1}}}
\expandafter\def\csname PY@tok@nn\endcsname{\let\PY@bf=\textbf\def\PY@tc##1{\textcolor[rgb]{0.00,0.00,1.00}{##1}}}
\expandafter\def\csname PY@tok@ne\endcsname{\let\PY@bf=\textbf\def\PY@tc##1{\textcolor[rgb]{0.82,0.25,0.23}{##1}}}
\expandafter\def\csname PY@tok@nv\endcsname{\def\PY@tc##1{\textcolor[rgb]{0.10,0.09,0.49}{##1}}}
\expandafter\def\csname PY@tok@no\endcsname{\def\PY@tc##1{\textcolor[rgb]{0.53,0.00,0.00}{##1}}}
\expandafter\def\csname PY@tok@nl\endcsname{\def\PY@tc##1{\textcolor[rgb]{0.63,0.63,0.00}{##1}}}
\expandafter\def\csname PY@tok@ni\endcsname{\let\PY@bf=\textbf\def\PY@tc##1{\textcolor[rgb]{0.60,0.60,0.60}{##1}}}
\expandafter\def\csname PY@tok@na\endcsname{\def\PY@tc##1{\textcolor[rgb]{0.49,0.56,0.16}{##1}}}
\expandafter\def\csname PY@tok@nt\endcsname{\let\PY@bf=\textbf\def\PY@tc##1{\textcolor[rgb]{0.00,0.50,0.00}{##1}}}
\expandafter\def\csname PY@tok@nd\endcsname{\def\PY@tc##1{\textcolor[rgb]{0.67,0.13,1.00}{##1}}}
\expandafter\def\csname PY@tok@s\endcsname{\def\PY@tc##1{\textcolor[rgb]{0.73,0.13,0.13}{##1}}}
\expandafter\def\csname PY@tok@sd\endcsname{\let\PY@it=\textit\def\PY@tc##1{\textcolor[rgb]{0.73,0.13,0.13}{##1}}}
\expandafter\def\csname PY@tok@si\endcsname{\let\PY@bf=\textbf\def\PY@tc##1{\textcolor[rgb]{0.73,0.40,0.53}{##1}}}
\expandafter\def\csname PY@tok@se\endcsname{\let\PY@bf=\textbf\def\PY@tc##1{\textcolor[rgb]{0.73,0.40,0.13}{##1}}}
\expandafter\def\csname PY@tok@sr\endcsname{\def\PY@tc##1{\textcolor[rgb]{0.73,0.40,0.53}{##1}}}
\expandafter\def\csname PY@tok@ss\endcsname{\def\PY@tc##1{\textcolor[rgb]{0.10,0.09,0.49}{##1}}}
\expandafter\def\csname PY@tok@sx\endcsname{\def\PY@tc##1{\textcolor[rgb]{0.00,0.50,0.00}{##1}}}
\expandafter\def\csname PY@tok@m\endcsname{\def\PY@tc##1{\textcolor[rgb]{0.40,0.40,0.40}{##1}}}
\expandafter\def\csname PY@tok@gh\endcsname{\let\PY@bf=\textbf\def\PY@tc##1{\textcolor[rgb]{0.00,0.00,0.50}{##1}}}
\expandafter\def\csname PY@tok@gu\endcsname{\let\PY@bf=\textbf\def\PY@tc##1{\textcolor[rgb]{0.50,0.00,0.50}{##1}}}
\expandafter\def\csname PY@tok@gd\endcsname{\def\PY@tc##1{\textcolor[rgb]{0.63,0.00,0.00}{##1}}}
\expandafter\def\csname PY@tok@gi\endcsname{\def\PY@tc##1{\textcolor[rgb]{0.00,0.63,0.00}{##1}}}
\expandafter\def\csname PY@tok@gr\endcsname{\def\PY@tc##1{\textcolor[rgb]{1.00,0.00,0.00}{##1}}}
\expandafter\def\csname PY@tok@ge\endcsname{\let\PY@it=\textit}
\expandafter\def\csname PY@tok@gs\endcsname{\let\PY@bf=\textbf}
\expandafter\def\csname PY@tok@gp\endcsname{\let\PY@bf=\textbf\def\PY@tc##1{\textcolor[rgb]{0.00,0.00,0.50}{##1}}}
\expandafter\def\csname PY@tok@go\endcsname{\def\PY@tc##1{\textcolor[rgb]{0.53,0.53,0.53}{##1}}}
\expandafter\def\csname PY@tok@gt\endcsname{\def\PY@tc##1{\textcolor[rgb]{0.00,0.27,0.87}{##1}}}
\expandafter\def\csname PY@tok@err\endcsname{\def\PY@bc##1{\setlength{\fboxsep}{0pt}\fcolorbox[rgb]{1.00,0.00,0.00}{1,1,1}{\strut ##1}}}
\expandafter\def\csname PY@tok@kc\endcsname{\let\PY@bf=\textbf\def\PY@tc##1{\textcolor[rgb]{0.00,0.50,0.00}{##1}}}
\expandafter\def\csname PY@tok@kd\endcsname{\let\PY@bf=\textbf\def\PY@tc##1{\textcolor[rgb]{0.00,0.50,0.00}{##1}}}
\expandafter\def\csname PY@tok@kn\endcsname{\let\PY@bf=\textbf\def\PY@tc##1{\textcolor[rgb]{0.00,0.50,0.00}{##1}}}
\expandafter\def\csname PY@tok@kr\endcsname{\let\PY@bf=\textbf\def\PY@tc##1{\textcolor[rgb]{0.00,0.50,0.00}{##1}}}
\expandafter\def\csname PY@tok@bp\endcsname{\def\PY@tc##1{\textcolor[rgb]{0.00,0.50,0.00}{##1}}}
\expandafter\def\csname PY@tok@fm\endcsname{\def\PY@tc##1{\textcolor[rgb]{0.00,0.00,1.00}{##1}}}
\expandafter\def\csname PY@tok@vc\endcsname{\def\PY@tc##1{\textcolor[rgb]{0.10,0.09,0.49}{##1}}}
\expandafter\def\csname PY@tok@vg\endcsname{\def\PY@tc##1{\textcolor[rgb]{0.10,0.09,0.49}{##1}}}
\expandafter\def\csname PY@tok@vi\endcsname{\def\PY@tc##1{\textcolor[rgb]{0.10,0.09,0.49}{##1}}}
\expandafter\def\csname PY@tok@vm\endcsname{\def\PY@tc##1{\textcolor[rgb]{0.10,0.09,0.49}{##1}}}
\expandafter\def\csname PY@tok@sa\endcsname{\def\PY@tc##1{\textcolor[rgb]{0.73,0.13,0.13}{##1}}}
\expandafter\def\csname PY@tok@sb\endcsname{\def\PY@tc##1{\textcolor[rgb]{0.73,0.13,0.13}{##1}}}
\expandafter\def\csname PY@tok@sc\endcsname{\def\PY@tc##1{\textcolor[rgb]{0.73,0.13,0.13}{##1}}}
\expandafter\def\csname PY@tok@dl\endcsname{\def\PY@tc##1{\textcolor[rgb]{0.73,0.13,0.13}{##1}}}
\expandafter\def\csname PY@tok@s2\endcsname{\def\PY@tc##1{\textcolor[rgb]{0.73,0.13,0.13}{##1}}}
\expandafter\def\csname PY@tok@sh\endcsname{\def\PY@tc##1{\textcolor[rgb]{0.73,0.13,0.13}{##1}}}
\expandafter\def\csname PY@tok@s1\endcsname{\def\PY@tc##1{\textcolor[rgb]{0.73,0.13,0.13}{##1}}}
\expandafter\def\csname PY@tok@mb\endcsname{\def\PY@tc##1{\textcolor[rgb]{0.40,0.40,0.40}{##1}}}
\expandafter\def\csname PY@tok@mf\endcsname{\def\PY@tc##1{\textcolor[rgb]{0.40,0.40,0.40}{##1}}}
\expandafter\def\csname PY@tok@mh\endcsname{\def\PY@tc##1{\textcolor[rgb]{0.40,0.40,0.40}{##1}}}
\expandafter\def\csname PY@tok@mi\endcsname{\def\PY@tc##1{\textcolor[rgb]{0.40,0.40,0.40}{##1}}}
\expandafter\def\csname PY@tok@il\endcsname{\def\PY@tc##1{\textcolor[rgb]{0.40,0.40,0.40}{##1}}}
\expandafter\def\csname PY@tok@mo\endcsname{\def\PY@tc##1{\textcolor[rgb]{0.40,0.40,0.40}{##1}}}
\expandafter\def\csname PY@tok@ch\endcsname{\let\PY@it=\textit\def\PY@tc##1{\textcolor[rgb]{0.25,0.50,0.50}{##1}}}
\expandafter\def\csname PY@tok@cm\endcsname{\let\PY@it=\textit\def\PY@tc##1{\textcolor[rgb]{0.25,0.50,0.50}{##1}}}
\expandafter\def\csname PY@tok@cpf\endcsname{\let\PY@it=\textit\def\PY@tc##1{\textcolor[rgb]{0.25,0.50,0.50}{##1}}}
\expandafter\def\csname PY@tok@c1\endcsname{\let\PY@it=\textit\def\PY@tc##1{\textcolor[rgb]{0.25,0.50,0.50}{##1}}}
\expandafter\def\csname PY@tok@cs\endcsname{\let\PY@it=\textit\def\PY@tc##1{\textcolor[rgb]{0.25,0.50,0.50}{##1}}}

\def\PYZbs{\char`\\}
\def\PYZus{\char`\_}
\def\PYZob{\char`\{}
\def\PYZcb{\char`\}}
\def\PYZca{\char`\^}
\def\PYZam{\char`\&}
\def\PYZlt{\char`\<}
\def\PYZgt{\char`\>}
\def\PYZsh{\char`\#}
\def\PYZpc{\char`\%}
\def\PYZdl{\char`\$}
\def\PYZhy{\char`\-}
\def\PYZsq{\char`\'}
\def\PYZdq{\char`\"}
\def\PYZti{\char`\~}
% for compatibility with earlier versions
\def\PYZat{@}
\def\PYZlb{[}
\def\PYZrb{]}
\makeatother


    % For linebreaks inside Verbatim environment from package fancyvrb. 
    \makeatletter
        \newbox\Wrappedcontinuationbox 
        \newbox\Wrappedvisiblespacebox 
        \newcommand*\Wrappedvisiblespace {\textcolor{red}{\textvisiblespace}} 
        \newcommand*\Wrappedcontinuationsymbol {\textcolor{red}{\llap{\tiny$\m@th\hookrightarrow$}}} 
        \newcommand*\Wrappedcontinuationindent {3ex } 
        \newcommand*\Wrappedafterbreak {\kern\Wrappedcontinuationindent\copy\Wrappedcontinuationbox} 
        % Take advantage of the already applied Pygments mark-up to insert 
        % potential linebreaks for TeX processing. 
        %        {, <, #, %, $, ' and ": go to next line. 
        %        _, }, ^, &, >, - and ~: stay at end of broken line. 
        % Use of \textquotesingle for straight quote. 
        \newcommand*\Wrappedbreaksatspecials {% 
            \def\PYGZus{\discretionary{\char`\_}{\Wrappedafterbreak}{\char`\_}}% 
            \def\PYGZob{\discretionary{}{\Wrappedafterbreak\char`\{}{\char`\{}}% 
            \def\PYGZcb{\discretionary{\char`\}}{\Wrappedafterbreak}{\char`\}}}% 
            \def\PYGZca{\discretionary{\char`\^}{\Wrappedafterbreak}{\char`\^}}% 
            \def\PYGZam{\discretionary{\char`\&}{\Wrappedafterbreak}{\char`\&}}% 
            \def\PYGZlt{\discretionary{}{\Wrappedafterbreak\char`\<}{\char`\<}}% 
            \def\PYGZgt{\discretionary{\char`\>}{\Wrappedafterbreak}{\char`\>}}% 
            \def\PYGZsh{\discretionary{}{\Wrappedafterbreak\char`\#}{\char`\#}}% 
            \def\PYGZpc{\discretionary{}{\Wrappedafterbreak\char`\%}{\char`\%}}% 
            \def\PYGZdl{\discretionary{}{\Wrappedafterbreak\char`\$}{\char`\$}}% 
            \def\PYGZhy{\discretionary{\char`\-}{\Wrappedafterbreak}{\char`\-}}% 
            \def\PYGZsq{\discretionary{}{\Wrappedafterbreak\textquotesingle}{\textquotesingle}}% 
            \def\PYGZdq{\discretionary{}{\Wrappedafterbreak\char`\"}{\char`\"}}% 
            \def\PYGZti{\discretionary{\char`\~}{\Wrappedafterbreak}{\char`\~}}% 
        } 
        % Some characters . , ; ? ! / are not pygmentized. 
        % This macro makes them "active" and they will insert potential linebreaks 
        \newcommand*\Wrappedbreaksatpunct {% 
            \lccode`\~`\.\lowercase{\def~}{\discretionary{\hbox{\char`\.}}{\Wrappedafterbreak}{\hbox{\char`\.}}}% 
            \lccode`\~`\,\lowercase{\def~}{\discretionary{\hbox{\char`\,}}{\Wrappedafterbreak}{\hbox{\char`\,}}}% 
            \lccode`\~`\;\lowercase{\def~}{\discretionary{\hbox{\char`\;}}{\Wrappedafterbreak}{\hbox{\char`\;}}}% 
            \lccode`\~`\:\lowercase{\def~}{\discretionary{\hbox{\char`\:}}{\Wrappedafterbreak}{\hbox{\char`\:}}}% 
            \lccode`\~`\?\lowercase{\def~}{\discretionary{\hbox{\char`\?}}{\Wrappedafterbreak}{\hbox{\char`\?}}}% 
            \lccode`\~`\!\lowercase{\def~}{\discretionary{\hbox{\char`\!}}{\Wrappedafterbreak}{\hbox{\char`\!}}}% 
            \lccode`\~`\/\lowercase{\def~}{\discretionary{\hbox{\char`\/}}{\Wrappedafterbreak}{\hbox{\char`\/}}}% 
            \catcode`\.\active
            \catcode`\,\active 
            \catcode`\;\active
            \catcode`\:\active
            \catcode`\?\active
            \catcode`\!\active
            \catcode`\/\active 
            \lccode`\~`\~ 	
        }
    \makeatother

    \let\OriginalVerbatim=\Verbatim
    \makeatletter
    \renewcommand{\Verbatim}[1][1]{%
        %\parskip\z@skip
        \sbox\Wrappedcontinuationbox {\Wrappedcontinuationsymbol}%
        \sbox\Wrappedvisiblespacebox {\FV@SetupFont\Wrappedvisiblespace}%
        \def\FancyVerbFormatLine ##1{\hsize\linewidth
            \vtop{\raggedright\hyphenpenalty\z@\exhyphenpenalty\z@
                \doublehyphendemerits\z@\finalhyphendemerits\z@
                \strut ##1\strut}%
        }%
        % If the linebreak is at a space, the latter will be displayed as visible
        % space at end of first line, and a continuation symbol starts next line.
        % Stretch/shrink are however usually zero for typewriter font.
        \def\FV@Space {%
            \nobreak\hskip\z@ plus\fontdimen3\font minus\fontdimen4\font
            \discretionary{\copy\Wrappedvisiblespacebox}{\Wrappedafterbreak}
            {\kern\fontdimen2\font}%
        }%
        
        % Allow breaks at special characters using \PYG... macros.
        \Wrappedbreaksatspecials
        % Breaks at punctuation characters . , ; ? ! and / need catcode=\active 	
        \OriginalVerbatim[#1,codes*=\Wrappedbreaksatpunct]%
    }
    \makeatother

    % Exact colors from NB
    \definecolor{incolor}{HTML}{303F9F}
    \definecolor{outcolor}{HTML}{D84315}
    \definecolor{cellborder}{HTML}{CFCFCF}
    \definecolor{cellbackground}{HTML}{F7F7F7}
    
    % prompt
    \makeatletter
    \newcommand{\boxspacing}{\kern\kvtcb@left@rule\kern\kvtcb@boxsep}
    \makeatother
    \newcommand{\prompt}[4]{
        {\ttfamily\llap{{\color{#2}[#3]:\hspace{3pt}#4}}\vspace{-\baselineskip}}
    }
    

    
    % Prevent overflowing lines due to hard-to-break entities
    \sloppy 
    % Setup hyperref package
    \hypersetup{
      breaklinks=true,  % so long urls are correctly broken across lines
      colorlinks=true,
      urlcolor=urlcolor,
      linkcolor=linkcolor,
      citecolor=citecolor,
      }
    % Slightly bigger margins than the latex defaults
    
    \geometry{verbose,tmargin=1in,bmargin=1in,lmargin=1in,rmargin=1in}

%%%%%%%%%% FIN DEL PREAMBULO DE NBCONVERT %%%%%%%%%%

\renewcommand{\tablename}{Tabla}
\renewcommand{\figurename}{Figura}
\renewcommand{\abstractname}{Resumen}
\renewcommand{\refname}{Referencias}
\let\originalcite\cite
\renewcommand{\cite}[2][]{\textsuperscript{\originalcite{#2}}}
% Cambiamos el estilo de las citas bibliograficas
\makeatletter\let\@afterindentfalse\@afterindenttrue\makeatother
% Para indentar primer parrafo
\setcounter{secnumdepth}{0}
% No se numeran las secciones pero si estan en el TOC
    
\title{Trabajo Práctico\\``Simulación de variables aleatorias''}
\author{Araneda, Alejandro – eloscurodeefeso@hotmail.com%
\and Quinteros, Fernando - lordfers@gmail.com%
\and Speciale, Gastón - gasticai@hotmail.com}
\date{2do. Cuatrimestre 2020\\Jueves, 3 de Diciembre}

\def\teacher{Dr. Ing. Néstor Rubén Barraza%
\and Dr. Lic. Verónica Moreno%
\and Ing. Gabriel Pena}

\begin{document}

\begin{titlepage}
\makeatletter
\renewcommand\and\par
\edef\oldparskip{\the\parskip}
\centering
\includegraphics{logo.png}\par
{\Large Ingeniería en Computación \par}\vspace{0.5cm}
{\LARGE Probabilidad y Estadística \par}\vfill
{\huge \@title \par}\vfill
Alumnos:\par\vspace{\oldparskip}
\setlength\parskip{0pt}\@author\setlength\parskip{\oldparskip}\vfill
Práctica entregada:\par
\@date\vfill
Docentes:\par\vspace{\oldparskip}
\setlength\parskip{0pt}\teacher\setlength\parskip{\oldparskip}\vspace{1cm}
\makeatother
\end{titlepage}

\begin{abstract}
    En el presente trabajo tiene el objetivo de servir de breve
    introducción a los métodos de generación de muestras de 
    variables aleatorias que siguen determinada distribución 
    de probabilidad.
\end{abstract}

\section{Introducción}

La inferencia de característica de sistemas complejos, así como
su modelado a través de técnicas numéricas, se ve beneficiado
si su verisimilitud es puesta a prueba mediante estímulos de 
naturaleza aleatoria\cite{bib:dek}. Siendo muchas veces necesario conducir
ese azar dentro del marco conceptual de los estudios de 
probabilidad y la estadística. De allí la búsqueda de métodos 
generadores de variables aleatorias con cierta distribución.

Uno de los métodos es la genaración da variables aleatorias 
por simulación directa. Este es caso para los ensayos Bernoulli,
cuyos resultados pueden ser sólo éxito o fracaso con cierta 
probabilidad. Sea $U$ una variable aleatoria con distribución
uniforme en el intervalo $[0, 1]$, entonces la simulación
directa de un ensayo Bernoulli con $p$ probabilidad de 
éxito viene dada por la función:

\[   
Ber(p) = 
     \begin{cases}
       1 &\quad U <= p\\
       0 &\quad\text{en otro caso}\\
     \end{cases}
\]

Con el mismo criterio podemos tambié generar una muestra de 
variables aleatorias que respeten una distribución binomial, 
define por la cantidad de éxitos de $n$ ensayos Bernoulli con 
$p$ probabilidad de éxito, mediante una suma directa.

Otro método es el uso de la inversa de la función densidad
de probabilidad acumulada. Es posible probar que si $F(x)$ es
la función densidad de probabilidad acumulada para algún 
valor $x$ de una variable aleatoria $X$ y $U$ es una variable
aleatoria uniformemente distribuida en el entorno
semiabierto $[0, 1)$,
entonces la función inversa $F^{-1}(U)$ nos permitirá obtener
una muestra de valores de $X$ que siguen su misma distribución
de probabilidad.

Así para la distribución exponencial con parámetro $\lambda$ 
igual a uno, su función densidad de probabilidad acumulada y 
su inversa son:

\[F(x) = 1 - e^{-x} \text{ con } x \geq 0\]
\[F^{-1}(u) = -ln(1 - u) = x\]

En el caso de la distribución normal, el cálculo analítico de
su inversa se ve imposibilitado. Es por ello que se emplea 
comúnmente un método que aproxima su inversa denominado 
Box-Müller por los nombres de sus desarrolladores. 
La técnica parte de suponer que $U_1$ y $U_2$ son dos variables
 aleatorias independientes que están uniformemente distribuidas
 en el intervalo semiabierto $(0, 1]$. Entonces $Z_0$ y $Z_1$ 
 son variables aleatorias independentes con una distribución 
 normal con desviación estandar igual a 1 y sus fórmulas son:

 \[Z_0 = R cos(\theta) = \sqrt{-2 ln(U_1)}cos(2\pi U_2)\]
 \[Z_1 = R sin(\theta) = \sqrt{-2 ln(U_1)}sin(2\pi U_2)\]

\section{Descripción de la práctica}

\subsection{Enunciado}

{\textbf{Parte 1: Simulación}}

En esta primera parte, construiremos varios generadores de números aleatorios que usaremos para obtener muestras con distribu-
ción conocida sobre las que vamos a trabajar posteriormente.
\begin{enumerate}
\item Utilizando únicamente la función random de su lenguaje (la función que genera un número aleatorio uniforme entre 0 y 1), implemente una función que genere un número distribuido Bernoulli con probabilidad p.
\item Utilizando la función del punto anterior, implemente otra que genere un número binomial con los parámetros n,p.
\item Utilizando el procedimiento descrito en el capítulo 6 del Dekking (método de la función inversa o de Monte Carlo), implementar una función que permita generar un número aleatorio con distribución Exp($\lambda$).
\item Investigar como generar números aleatorios con distribución normal. Implementarlo.
\end{enumerate}

{\textbf{Parte 2: Estadística descriptiva}}

Ahora vamos a aplicar las técnicas vistas en la materia al estudio de algunas muestras de datos.
\begin{enumerate}
\item Generar tres muestras de números aleatorios Exp(0,5) de tamaño n = 10, n = 30 y n = 200. Para cada una, computar la media y varianza muestral. ¿Qué observa?
\item Para las tres muestras anteriores, graficar los histogramas de frecuencias relativas con anchos de banda 0,4, 0,2 y 0,1; es decir, un total de 9 histogramas. ¿Qué conclusiones puede obtener?
\item Generar una muestra de números Bin(10,0,3) de tamaño n = 50. Construir la función de distribución empírica de dicha muestra.
\item A partir de la función de distribución empírica del punto anterior, generar una nueva muestra de números aleatorios utilizando el método de simulación de la primera parte. Computar la media y varianza muestral y graficar el histograma.
\item Repetir el experimento de los dos puntos anteriores con dos muestras aleatorias más generadas con los mismos parámetros. ¿Qué conclusión saca?
\end{enumerate}

{\textbf{Parte 3: Convergencia}}

El propósito de esta sección es ver en forma práctica los resultados de los teoremas de convergencia.
\begin{enumerate}
\item Generar cuatro muestras de números aleatorios de tamaño 100, todas con distribución binomial con p = 0,40 y n = 10, n = 20, n = 50 y n = 100 respectivamente. Graficar sus histogramas. ¿Qué observa?
\item Elija la muestra de tamaño 200 y calcule la media y desviación estándar muestral. Luego, normalice cada dato de la muestra y grafique el histograma de la muestra normalizada. Justifique lo que observa.
\item Para cada una de las muestras anteriores, calcule la media muestral. Justifique lo que observa.
\end{enumerate}

{\textbf{Parte 4: Estadística inferencial}}

Para terminar, vamos a hacer inferencia con las muestras que generamos y obtener así información sobre sus distribuciones.
\begin{enumerate}
\item Generar dos muestras N(100,5), una de tamaño n = 10 y otra de tamaño n = 30. Obtener estimaciones puntuales de su media y varianza.
\item Suponga que ya conoce el dato de que la distribución tiene varianza 5. Obtener intervalos de confianza del 95% y 98% para la media de ambas muestras.
\item Repita el punto anterior pero usando la varianza estimada s2, para la muestra de tamaño adecuado.
\item Probar a nivel 0,99 la hipótesis de que la varianza sea $\sigma^2$ > 5. Calcular la probabilidad de cometer error tipo II para la hipótesis alternativa $\sigma^2$ = 6.
\item Agrupando los datos en subgrupos de longitud 0,5, probar a nivel 0,99 la hipótesis de que la muestra proviene de una distribución normal.
\end{enumerate}

\subsection{Entorno de Desarrollo}

Siguiendo las indicaciones de la cátedra, implementamos los 
generadores de muestras aleatorias con el lenguaje Python. 
Nuestro desarrollo tiene dependencia de librerías del 
proyecto \textit{SciPy}: los ejercicios son resueltos
en el entorno interactivo \textit{Jupyter Notebook}\cite{bib:jupyter},
las gráficas generadas con \textit{Matplotlib}\cite{bib:matplot} y
las inversas de las tablas de la normal estandariza, la 
distribució T de Studente y la $\chi^2$ son tomadas de la
librería \textit{Scipy}\cite{bib:scipylib}.

Para documentar el código fuente nos adherimos a las 
convenciones del lenguaje utilizando \textit{Docstring}\cite{bib:doc}.
De esta forma, su presentación la automatizamos mediante el 
paquete provisto por el proyecto \textit{Sphinx}\cite{bib:sphinx}. Para la 
presetación final del trabajo, las \textit{notebooks} se 
unifican mediante el paquete \textit{nbmerge}\cite{bib:merge} he 
integradas manualmente a este informe para ajustes estéticos.

Las fuciones principales de cada módulo y que nos sirven en la 
resolución de los ejercicios propuestos, han sido verificadas
en su funcionamiento previamente mediante el uso de la librería 
\textit{Unittest}\cite{bib:test}.

La organización del aspecto colaborativo entre los participantes
de la práctica es resuelto mediante la utilización de un 
repositorio de cotrol de versiones alojado en el sitio 
\textit{GitHub}\cite{bib:git}.

\section{Parte 1: Simulación}

Todo el código desarrollado se encuentra profusamente documentado
en un manual anexo. Sin embargo, utilizaremos la presente
sección para describir el contenido de los archivos de la 
práctica que son entregados juntamente con este informe.

El uso del entorno interactivo correspondiente al proyecto
\textit{Jupiter Notebook}, nos decidió por una arquitectura 
simple con módulos de funciones estáticas separados por 
incumbencias.

\begin{table}[]
    \begin{tabular}{|l|p{0.5\paperwidth}|}
    \hline
    \textbf{Archivo}                 & \textbf{Descripción}                                                                            \\ \hline
    {\ttfamily analisis.py}                      & Módulo python la implementación de las funciones que calculan las medias y varianzas muestrales \\ \hline
    {\ttfamily build\textbackslash{}mytitle.sty} & Paquete de latex con la carátula del manual                                                     \\ \hline
    {\ttfamily build\textbackslash{}manual.pdf}  & El manual con la documentación autogenerada de todo el código                                   \\ \hline
    {\ttfamily graficas.py}                      & Módulo python auxiliar para la generación de gráficas y tablas de la práctica                   \\ \hline
    {\ttfamily informe.tex}                      & Fuente latex del presente informe                                                               \\ \hline
    {\ttfamily informe.pdf}                      & Este informe                                                                                    \\ \hline
    {\ttfamily logo.png}                         & Logo de la Universidad Nacional de Tres de Febrero                                              \\ \hline
    {\ttfamily parte2.ipynb}                     & Parte 2 de la práctica desarrollada en una Jupyter Notebook                                     \\ \hline
    {\ttfamily parte3.ipynb}                     & Parte 3 de la práctica desarrollada en una Jupyter Notebook                                     \\ \hline
    {\ttfamily parte4.ipynb}                     & Parte 4 de la práctica desarrollada en una Jupyter Notebook                                     \\ \hline
    {\ttfamily simulador.py}                     & Módulo python con la implemetación de los generadores de muestras aleatorias                    \\ \hline
    {\ttfamily source\textbackslash{}conf.py}    & Archivo de configuración para extracción de la documentación del código fuente                  \\ \hline
    {\ttfamily source\textbackslash{}index.rst}  & Archivo de configuración para la extracción de la documentación del código fuente               \\ \hline
    \end{tabular}
    \end{table}

%%%%%%%%%% INSERTAR ACA LA SALIDA DE NBCONVERT %%%%%%%%%%
    
    
    \hypertarget{parte-2-estaduxedstica-descriptiva}{%
\section{Parte 2: Estadística
descriptiva}\label{parte-2-estaduxedstica-descriptiva}}

Primeramente es necesario importar las funciones auxiliares que hemos
implementado para simular las muestras de variables aleatorias:

    \begin{tcolorbox}[breakable, size=fbox, boxrule=1pt, pad at break*=1mm,colback=cellbackground, colframe=cellborder]
\prompt{In}{incolor}{1}{\boxspacing}
\begin{Verbatim}[commandchars=\\\{\}]
\PY{k+kn}{import} \PY{n+nn}{sys}\PY{p}{;} \PY{n}{sys}\PY{o}{.}\PY{n}{path}\PY{o}{.}\PY{n}{insert}\PY{p}{(}\PY{l+m+mi}{0}\PY{p}{,} \PY{l+s+s1}{\PYZsq{}}\PY{l+s+s1}{..}\PY{l+s+s1}{\PYZsq{}}\PY{p}{)}
\PY{k+kn}{from} \PY{n+nn}{simulador} \PY{k+kn}{import} \PY{o}{*}
\end{Verbatim}
\end{tcolorbox}

    Luego obtenemos tres muestras de números aleatorios con distribución
exponencial y parámetro lambda (la inversa de la media) igual a 0,5 pero
de diferentes tamaños: 10, 30 y 200 datos cada una. Las llamaremos
acordemente \texttt{exp10}, \texttt{exp30} y \texttt{exp200}
respectivamente:

    \begin{tcolorbox}[breakable, size=fbox, boxrule=1pt, pad at break*=1mm,colback=cellbackground, colframe=cellborder]
\prompt{In}{incolor}{2}{\boxspacing}
\begin{Verbatim}[commandchars=\\\{\}]
\PY{n}{exp10} \PY{o}{=} \PY{n}{exponencial}\PY{p}{(}\PY{l+m+mi}{10}\PY{p}{,} \PY{l+m+mi}{2}\PY{p}{)}
\PY{n}{exp30} \PY{o}{=} \PY{n}{exponencial}\PY{p}{(}\PY{l+m+mi}{30}\PY{p}{,} \PY{l+m+mi}{2}\PY{p}{)}
\PY{n}{exp200} \PY{o}{=} \PY{n}{exponencial}\PY{p}{(}\PY{l+m+mi}{200}\PY{p}{,} \PY{l+m+mi}{2}\PY{p}{)}
\end{Verbatim}
\end{tcolorbox}

    Ahora importaremos las funciones \texttt{media\_muestral} y
\texttt{varianza\_muestral} del módulo \texttt{analisis} para completar
los datos de la próxima tabla.

    \begin{tcolorbox}[breakable, size=fbox, boxrule=1pt, pad at break*=1mm,colback=cellbackground, colframe=cellborder]
\prompt{In}{incolor}{3}{\boxspacing}
\begin{Verbatim}[commandchars=\\\{\}]
\PY{k+kn}{from} \PY{n+nn}{analisis} \PY{k+kn}{import} \PY{n}{media\PYZus{}muestral} \PY{k}{as} \PY{n}{mm}
\PY{k+kn}{from} \PY{n+nn}{analisis} \PY{k+kn}{import} \PY{n}{varianza\PYZus{}muestral} \PY{k}{as} \PY{n}{vm}
\end{Verbatim}
\end{tcolorbox}

    \begin{tcolorbox}[breakable, size=fbox, boxrule=1pt, pad at break*=1mm,colback=cellbackground, colframe=cellborder]
\prompt{In}{incolor}{4}{\boxspacing}
\begin{Verbatim}[commandchars=\\\{\}]
\PY{k+kn}{from} \PY{n+nn}{IPython}\PY{n+nn}{.}\PY{n+nn}{display} \PY{k+kn}{import} \PY{n}{display}\PY{p}{,} \PY{n}{Markdown}
\PY{n}{display}\PY{p}{(}\PY{n}{Markdown}\PY{p}{(}\PY{l+s+s1}{\PYZsq{}\PYZsq{}\PYZsq{}}
\PY{l+s+s1}{| Tamaño | Media | Varianza |}
\PY{l+s+s1}{|\PYZhy{}\PYZhy{}\PYZhy{}\PYZhy{}\PYZhy{}\PYZhy{}\PYZhy{}\PYZhy{}|\PYZhy{}\PYZhy{}\PYZhy{}\PYZhy{}\PYZhy{}\PYZhy{}\PYZhy{}|\PYZhy{}\PYZhy{}\PYZhy{}\PYZhy{}\PYZhy{}\PYZhy{}\PYZhy{}\PYZhy{}\PYZhy{}\PYZhy{}|}
\PY{l+s+s1}{| }\PY{l+s+si}{\PYZpc{}d}\PY{l+s+s1}{     | }\PY{l+s+si}{\PYZpc{}.2f}\PY{l+s+s1}{  | }\PY{l+s+si}{\PYZpc{}.2f}\PY{l+s+s1}{     |}
\PY{l+s+s1}{| }\PY{l+s+si}{\PYZpc{}d}\PY{l+s+s1}{     | }\PY{l+s+si}{\PYZpc{}.2f}\PY{l+s+s1}{  | }\PY{l+s+si}{\PYZpc{}.2f}\PY{l+s+s1}{     |}
\PY{l+s+s1}{| }\PY{l+s+si}{\PYZpc{}d}\PY{l+s+s1}{     | }\PY{l+s+si}{\PYZpc{}.2f}\PY{l+s+s1}{  | }\PY{l+s+si}{\PYZpc{}.2f}\PY{l+s+s1}{     |}
\PY{l+s+s1}{\PYZsq{}\PYZsq{}\PYZsq{}} \PY{o}{\PYZpc{}} \PY{p}{(}\PY{l+m+mi}{10}\PY{p}{,} \PY{n}{mm}\PY{p}{(}\PY{n}{exp10}\PY{p}{)}\PY{p}{,} \PY{n}{vm}\PY{p}{(}\PY{n}{exp10}\PY{p}{)}\PY{p}{,}
       \PY{l+m+mi}{30}\PY{p}{,} \PY{n}{mm}\PY{p}{(}\PY{n}{exp30}\PY{p}{)}\PY{p}{,} \PY{n}{vm}\PY{p}{(}\PY{n}{exp30}\PY{p}{)}\PY{p}{,}
       \PY{l+m+mi}{200}\PY{p}{,} \PY{n}{mm}\PY{p}{(}\PY{n}{exp200}\PY{p}{)}\PY{p}{,} \PY{n}{vm}\PY{p}{(}\PY{n}{exp200}\PY{p}{)}\PY{p}{)}\PY{p}{)}\PY{p}{)}
\end{Verbatim}
\end{tcolorbox}

    \begin{longtable}[]{@{}lll@{}}
\toprule
Tamaño & Media & Varianza\tabularnewline
\midrule
\endhead
10 & 2.20 & 2.32\tabularnewline
30 & 2.84 & 9.56\tabularnewline
200 & 1.80 & 3.03\tabularnewline
\bottomrule
\end{longtable}

    
    Lo que podemos concluir al observar la repetición de la experiencia es
que si bien como es esperable los parámetros de la muestra se aproximan
a los de la distribución cuando el tamaño de la muestra crece, la media
muestral lo hace con respecto a la media de la distribución con mayor
estabilidad que la varianza muestral con respecto a la varianza de la
distribución.

Observaremos a continuación los histogramas de las muestras con
diferente anchos de banda, para lo cual importamos una función auxiliar
que hace uso de la librería gráfica elegida. Presentamos a continuación
los histogramas con anchos de banda de 0,4 unidades (izquierda), de 0,2
unidades (centro) y de 0,1 unidades (derecha) para nuestra muestra de 10
datos.

    \begin{tcolorbox}[breakable, size=fbox, boxrule=1pt, pad at break*=1mm,colback=cellbackground, colframe=cellborder]
\prompt{In}{incolor}{5}{\boxspacing}
\begin{Verbatim}[commandchars=\\\{\}]
\PY{o}{\PYZpc{}}\PY{k}{matplotlib} inline
\PY{k+kn}{from} \PY{n+nn}{graficas} \PY{k+kn}{import} \PY{n}{histograma}
\PY{n}{histograma}\PY{p}{(}\PY{n}{exp10}\PY{p}{,} \PY{p}{[}\PY{l+m+mf}{0.4}\PY{p}{,}\PY{l+m+mf}{0.2}\PY{p}{,}\PY{l+m+mf}{0.1}\PY{p}{]}\PY{p}{,} \PY{n}{relative}\PY{o}{=}\PY{k+kc}{True}\PY{p}{)}
\end{Verbatim}
\end{tcolorbox}

    \begin{center}
    \adjustimage{max size={0.9\linewidth}{0.9\paperheight}}{output_files/output_8_0.png}
    \end{center}
    { \hspace*{\fill} \\}
    
    Reiteramos el procedimiento con la muestra de 30 datos:

    \begin{tcolorbox}[breakable, size=fbox, boxrule=1pt, pad at break*=1mm,colback=cellbackground, colframe=cellborder]
\prompt{In}{incolor}{6}{\boxspacing}
\begin{Verbatim}[commandchars=\\\{\}]
\PY{n}{histograma}\PY{p}{(}\PY{n}{exp30}\PY{p}{,} \PY{p}{[}\PY{l+m+mf}{0.4}\PY{p}{,}\PY{l+m+mf}{0.2}\PY{p}{,}\PY{l+m+mf}{0.1}\PY{p}{]}\PY{p}{,} \PY{n}{relative}\PY{o}{=}\PY{k+kc}{True}\PY{p}{)}
\end{Verbatim}
\end{tcolorbox}

    \begin{center}
    \adjustimage{max size={0.9\linewidth}{0.9\paperheight}}{output_files/output_10_0.png}
    \end{center}
    { \hspace*{\fill} \\}
    
    Y finalmente, obtenemos los histogramas para nuestra muestra de 200
variables aleatorias en base a una distribución exponencial con los
mismos parámetros:

    \begin{tcolorbox}[breakable, size=fbox, boxrule=1pt, pad at break*=1mm,colback=cellbackground, colframe=cellborder]
\prompt{In}{incolor}{7}{\boxspacing}
\begin{Verbatim}[commandchars=\\\{\}]
\PY{n}{histograma}\PY{p}{(}\PY{n}{exp200}\PY{p}{,} \PY{p}{[}\PY{l+m+mf}{0.4}\PY{p}{,}\PY{l+m+mf}{0.2}\PY{p}{,}\PY{l+m+mf}{0.1}\PY{p}{]}\PY{p}{,} \PY{n}{relative}\PY{o}{=}\PY{k+kc}{True}\PY{p}{)}
\end{Verbatim}
\end{tcolorbox}

    \begin{center}
    \adjustimage{max size={0.9\linewidth}{0.9\paperheight}}{output_files/output_12_0.png}
    \end{center}
    { \hspace*{\fill} \\}
    
    El análisis de los histogramas nos permite verificar que nuestras
muestras efectivamente se aproximan a la distribución exponencial cuando
mayor es el tamaño de la muestra. También podemos notar que no todos los
anchos de banda nos brindan la misma información: para una muestra de
tamaño discreto, un ancho de banda muy pequeño podría ubicar quizá sólo
una aparición por clase, de manera que las características de nuestra
distribución se hallen oscurecidas.

Lo siguiente que haremos será crear una muestra de 50 números con base
en una distribución binomial con parámetros correspondientes a 10
experimentos Bernoulli y una probabilidad de éxito de 0,3. También
graficaremos su histograma como referencia a la distribución empírica
que producirá e informaremos sus estadísticos con la función auxiliar
\texttt{estadist} importada del módulo \texttt{analisis.py}.

    \begin{tcolorbox}[breakable, size=fbox, boxrule=1pt, pad at break*=1mm,colback=cellbackground, colframe=cellborder]
\prompt{In}{incolor}{8}{\boxspacing}
\begin{Verbatim}[commandchars=\\\{\}]
\PY{k+kn}{from} \PY{n+nn}{analisis} \PY{k+kn}{import} \PY{n}{estadist}
\PY{n}{bin50} \PY{o}{=} \PY{n}{binomial}\PY{p}{(}\PY{l+m+mi}{50}\PY{p}{,} \PY{l+m+mi}{10}\PY{p}{,} \PY{l+m+mf}{0.3}\PY{p}{)}
\PY{n}{estadist}\PY{p}{(}\PY{n}{bin50}\PY{p}{)}
\PY{n}{histograma}\PY{p}{(}\PY{n}{bin50}\PY{p}{,} \PY{p}{[}\PY{l+m+mi}{1}\PY{p}{]}\PY{p}{)}
\end{Verbatim}
\end{tcolorbox}

    \begin{Verbatim}[commandchars=\\\{\}]
Media muestral: 2.94 - Varianza muestral: 1.85
    \end{Verbatim}

    \begin{center}
    \adjustimage{max size={0.9\linewidth}{0.9\paperheight}}{output_files/output_14_1.png}
    \end{center}
    { \hspace*{\fill} \\}
    
    Ahora simulamos una segunda muestra de 50 dotas que tendrán por base la
distribución empírica que resulta de la muestra anterior. Calculamos
también su media y varianza muestral además de graficar el histograma.

    \begin{tcolorbox}[breakable, size=fbox, boxrule=1pt, pad at break*=1mm,colback=cellbackground, colframe=cellborder]
\prompt{In}{incolor}{9}{\boxspacing}
\begin{Verbatim}[commandchars=\\\{\}]
\PY{n}{bino} \PY{o}{=} \PY{n}{binomial}\PY{p}{(}\PY{l+m+mi}{50}\PY{p}{,} \PY{l+m+mi}{10}\PY{p}{,} \PY{l+m+mf}{0.3}\PY{p}{)}
\PY{n}{emp} \PY{o}{=} \PY{n}{empirica}\PY{p}{(}\PY{l+m+mi}{50}\PY{p}{,} \PY{n}{bino}\PY{p}{)}
\PY{n+nb}{print}\PY{p}{(}\PY{l+s+s2}{\PYZdq{}}\PY{l+s+s2}{Muestra binomial \PYZhy{} }\PY{l+s+s2}{\PYZdq{}}\PY{p}{,} \PY{n}{end}\PY{o}{=}\PY{l+s+s2}{\PYZdq{}}\PY{l+s+s2}{\PYZdq{}}\PY{p}{)}
\PY{n}{estadist}\PY{p}{(}\PY{n}{bino}\PY{p}{)}
\PY{n}{histograma}\PY{p}{(}\PY{n}{bino}\PY{p}{,} \PY{p}{[}\PY{l+m+mi}{1}\PY{p}{]}\PY{p}{)}
\PY{n+nb}{print}\PY{p}{(}\PY{l+s+s2}{\PYZdq{}}\PY{l+s+s2}{Muestra empírica \PYZhy{} }\PY{l+s+s2}{\PYZdq{}}\PY{p}{,} \PY{n}{end}\PY{o}{=}\PY{l+s+s2}{\PYZdq{}}\PY{l+s+s2}{\PYZdq{}}\PY{p}{)}
\PY{n}{estadist}\PY{p}{(}\PY{n}{emp}\PY{p}{)}
\PY{n}{histograma}\PY{p}{(}\PY{n}{emp}\PY{p}{,} \PY{p}{[}\PY{l+m+mi}{1}\PY{p}{]}\PY{p}{)}
\end{Verbatim}
\end{tcolorbox}

    \begin{Verbatim}[commandchars=\\\{\}]
Media muestral: 3.22 - Varianza muestral: 1.89
    \end{Verbatim}

    \begin{center}
    \adjustimage{max size={0.9\linewidth}{0.9\paperheight}}{output_files/output_16_1.png}
    \end{center}
    { \hspace*{\fill} \\}
    
    \begin{Verbatim}[commandchars=\\\{\}]
Muestra empírica - Media muestral: 2.60 - Varianza muestral: 1.47
    \end{Verbatim}

    \begin{center}
    \adjustimage{max size={0.9\linewidth}{0.9\paperheight}}{output_files/output_16_3.png}
    \end{center}
    { \hspace*{\fill} \\}
    
    \begin{tcolorbox}[breakable, size=fbox, boxrule=1pt, pad at break*=1mm,colback=cellbackground, colframe=cellborder]
\prompt{In}{incolor}{11}{\boxspacing}
\begin{Verbatim}[commandchars=\\\{\}]
\PY{n}{bino} \PY{o}{=} \PY{n}{binomial}\PY{p}{(}\PY{l+m+mi}{50}\PY{p}{,} \PY{l+m+mi}{10}\PY{p}{,} \PY{l+m+mf}{0.3}\PY{p}{)}
\PY{n}{emp} \PY{o}{=} \PY{n}{empirica}\PY{p}{(}\PY{l+m+mi}{50}\PY{p}{,} \PY{n}{bino}\PY{p}{)}
\PY{n+nb}{print}\PY{p}{(}\PY{l+s+s2}{\PYZdq{}}\PY{l+s+s2}{Muestra binomial \PYZhy{} }\PY{l+s+s2}{\PYZdq{}}\PY{p}{,} \PY{n}{end}\PY{o}{=}\PY{l+s+s2}{\PYZdq{}}\PY{l+s+s2}{\PYZdq{}}\PY{p}{)}
\PY{n}{estadist}\PY{p}{(}\PY{n}{bino}\PY{p}{)}
\PY{n}{histograma}\PY{p}{(}\PY{n}{bino}\PY{p}{,} \PY{p}{[}\PY{l+m+mi}{1}\PY{p}{]}\PY{p}{)}
\PY{n+nb}{print}\PY{p}{(}\PY{l+s+s2}{\PYZdq{}}\PY{l+s+s2}{Muestra empírica \PYZhy{} }\PY{l+s+s2}{\PYZdq{}}\PY{p}{,} \PY{n}{end}\PY{o}{=}\PY{l+s+s2}{\PYZdq{}}\PY{l+s+s2}{\PYZdq{}}\PY{p}{)}
\PY{n}{estadist}\PY{p}{(}\PY{n}{emp}\PY{p}{)}
\PY{n}{histograma}\PY{p}{(}\PY{n}{emp}\PY{p}{,} \PY{p}{[}\PY{l+m+mi}{1}\PY{p}{]}\PY{p}{)}
\end{Verbatim}
\end{tcolorbox}

    \begin{Verbatim}[commandchars=\\\{\}]
Muestra binomial - Media muestral: 2.86 - Varianza muestral: 1.84
    \end{Verbatim}

    \begin{center}
    \adjustimage{max size={0.9\linewidth}{0.9\paperheight}}{output_files/output_17_1.png}
    \end{center}
    { \hspace*{\fill} \\}
    
    \begin{Verbatim}[commandchars=\\\{\}]
Muestra empírica - Media muestral: 2.80 - Varianza muestral: 2.41
    \end{Verbatim}

    \begin{center}
    \adjustimage{max size={0.9\linewidth}{0.9\paperheight}}{output_files/output_17_3.png}
    \end{center}
    { \hspace*{\fill} \\}
    
    Como conclusión podemos decir que si bien las muestras binomiales siguen
una implementación que tiende a aproximar sus estadísticos a los que se
esperarían en base a los parámetros de la distribución binomial de base,
lo hace de una manera que a simple observación parece más azarosa que la
que resulta de obtener nuevas muestras utilizándolas de distribución
empírica.

    \hypertarget{parte-3-convergencia}{%
\section{Parte 3: Convergencia}\label{parte-3-convergencia}}

Al igual que en la parte 2, es necesario importar las funciones
auxiliares que hemos implementado para simular las muestras de variables
aleatorias:

    \begin{tcolorbox}[breakable, size=fbox, boxrule=1pt, pad at break*=1mm,colback=cellbackground, colframe=cellborder]
\prompt{In}{incolor}{1}{\boxspacing}
\begin{Verbatim}[commandchars=\\\{\}]
\PY{k+kn}{import} \PY{n+nn}{sys}\PY{p}{;} \PY{n}{sys}\PY{o}{.}\PY{n}{path}\PY{o}{.}\PY{n}{insert}\PY{p}{(}\PY{l+m+mi}{0}\PY{p}{,} \PY{l+s+s1}{\PYZsq{}}\PY{l+s+s1}{..}\PY{l+s+s1}{\PYZsq{}}\PY{p}{)}
\PY{k+kn}{from} \PY{n+nn}{simulador} \PY{k+kn}{import} \PY{o}{*}
\end{Verbatim}
\end{tcolorbox}

    Luego obtenemos cuatro muestras con distribucion binomial de tamaño 100
con probabilidad p igual a 0.4, pero cada una con 10, 20, 50 y 100
experimentos Bernoulli, que se denotaran como \texttt{bin10},
\texttt{bin20}, \texttt{bin50} y \texttt{bin100} respectivamente:

    \begin{tcolorbox}[breakable, size=fbox, boxrule=1pt, pad at break*=1mm,colback=cellbackground, colframe=cellborder]
\prompt{In}{incolor}{2}{\boxspacing}
\begin{Verbatim}[commandchars=\\\{\}]
\PY{n}{bin10} \PY{o}{=} \PY{n}{binomial}\PY{p}{(}\PY{l+m+mi}{100}\PY{p}{,} \PY{l+m+mi}{10}\PY{p}{,} \PY{l+m+mf}{0.4}\PY{p}{)}
\PY{n}{bin20} \PY{o}{=} \PY{n}{binomial}\PY{p}{(}\PY{l+m+mi}{100}\PY{p}{,} \PY{l+m+mi}{20}\PY{p}{,} \PY{l+m+mf}{0.4}\PY{p}{)}
\PY{n}{bin50} \PY{o}{=} \PY{n}{binomial}\PY{p}{(}\PY{l+m+mi}{100}\PY{p}{,} \PY{l+m+mi}{50}\PY{p}{,} \PY{l+m+mf}{0.4}\PY{p}{)}
\PY{n}{bin100} \PY{o}{=} \PY{n}{binomial}\PY{p}{(}\PY{l+m+mi}{100}\PY{p}{,} \PY{l+m+mi}{100}\PY{p}{,} \PY{l+m+mf}{0.4}\PY{p}{)}
\end{Verbatim}
\end{tcolorbox}

    Ahora, procedemos a graficar los histogramas correspondientes a cada
muestra de datos obtenida.

Para una muestra binomial con 10 experimentos Bernoulli, probabilidad
0.4 y con 100 repeticiones, el histograma es el siguiente:

    \begin{tcolorbox}[breakable, size=fbox, boxrule=1pt, pad at break*=1mm,colback=cellbackground, colframe=cellborder]
\prompt{In}{incolor}{3}{\boxspacing}
\begin{Verbatim}[commandchars=\\\{\}]
\PY{o}{\PYZpc{}}\PY{k}{matplotlib} inline
\PY{k+kn}{from} \PY{n+nn}{graficas} \PY{k+kn}{import} \PY{n}{histograma}
\PY{n}{histograma}\PY{p}{(}\PY{n}{bin10}\PY{p}{,} \PY{p}{[}\PY{l+m+mi}{1}\PY{p}{]}\PY{p}{)}
\end{Verbatim}
\end{tcolorbox}

    \begin{center}
    \adjustimage{max size={0.9\linewidth}{0.9\paperheight}}{output_files/output_24_0.png}
    \end{center}
    { \hspace*{\fill} \\}
    
    Para una muestra binomial con 20 experimentos Bernoulli, probabilidad
0.4 y con 100 repeticiones, el histograma es el siguiente:

    \begin{tcolorbox}[breakable, size=fbox, boxrule=1pt, pad at break*=1mm,colback=cellbackground, colframe=cellborder]
\prompt{In}{incolor}{4}{\boxspacing}
\begin{Verbatim}[commandchars=\\\{\}]
\PY{n}{histograma}\PY{p}{(}\PY{n}{bin20}\PY{p}{,} \PY{p}{[}\PY{l+m+mi}{1}\PY{p}{]}\PY{p}{)}
\end{Verbatim}
\end{tcolorbox}

    \begin{center}
    \adjustimage{max size={0.9\linewidth}{0.9\paperheight}}{output_files/output_26_0.png}
    \end{center}
    { \hspace*{\fill} \\}
    
    Para una muestra binomial con 50 experimentos Bernoulli, probabilidad
0.4 y con 100 repeticiones, el histograma es el siguiente:

    \begin{tcolorbox}[breakable, size=fbox, boxrule=1pt, pad at break*=1mm,colback=cellbackground, colframe=cellborder]
\prompt{In}{incolor}{5}{\boxspacing}
\begin{Verbatim}[commandchars=\\\{\}]
\PY{n}{histograma}\PY{p}{(}\PY{n}{bin50}\PY{p}{,} \PY{p}{[}\PY{l+m+mi}{1}\PY{p}{]}\PY{p}{)}
\end{Verbatim}
\end{tcolorbox}

    \begin{center}
    \adjustimage{max size={0.9\linewidth}{0.9\paperheight}}{output_files/output_28_0.png}
    \end{center}
    { \hspace*{\fill} \\}
    
    Para una muestra binomial con 100 experimentos Bernoulli, probabilidad
0.4 y con 100 repeticiones, el histograma es el siguiente:

    \begin{tcolorbox}[breakable, size=fbox, boxrule=1pt, pad at break*=1mm,colback=cellbackground, colframe=cellborder]
\prompt{In}{incolor}{6}{\boxspacing}
\begin{Verbatim}[commandchars=\\\{\}]
\PY{n}{histograma}\PY{p}{(}\PY{n}{bin100}\PY{p}{,} \PY{p}{[}\PY{l+m+mi}{1}\PY{p}{]}\PY{p}{)}
\end{Verbatim}
\end{tcolorbox}

    \begin{center}
    \adjustimage{max size={0.9\linewidth}{0.9\paperheight}}{output_files/output_30_0.png}
    \end{center}
    { \hspace*{\fill} \\}
    
    Lo que se puede observar es que a medida de que la cantidad de
experimentos de la muestra crece, es decir n crece, la muestra empieza a
tener la forma de una campana de Gauss y que se aproxima a una muestra
de distribucion normal. Si se observa desde el primer histograma cuyo n
es 10, hasta el ultimo cuyo n es 100, la forma de campana se hace mas
notoria.

Ahora procedemos a graficar una muestra de tamaño 100 con probabilidad
de éxito 0.4 para 200 experimentos Bernoulli. Calcularemos también su
media y desviacion estandar muestrales.

    \begin{tcolorbox}[breakable, size=fbox, boxrule=1pt, pad at break*=1mm,colback=cellbackground, colframe=cellborder]
\prompt{In}{incolor}{7}{\boxspacing}
\begin{Verbatim}[commandchars=\\\{\}]
\PY{k+kn}{from} \PY{n+nn}{analisis} \PY{k+kn}{import} \PY{n}{media\PYZus{}muestral} \PY{k}{as} \PY{n}{mm}
\PY{k+kn}{from} \PY{n+nn}{analisis} \PY{k+kn}{import} \PY{n}{desviacion\PYZus{}estandar\PYZus{}muestral} \PY{k}{as} \PY{n}{dem}
\PY{n}{bin200} \PY{o}{=} \PY{n}{binomial}\PY{p}{(}\PY{l+m+mi}{100}\PY{p}{,} \PY{l+m+mi}{200}\PY{p}{,} \PY{l+m+mf}{0.4}\PY{p}{)}
\PY{n}{bin200\PYZus{}mm} \PY{o}{=} \PY{n}{mm}\PY{p}{(}\PY{n}{bin200}\PY{p}{)}
\PY{n}{bin200\PYZus{}dem} \PY{o}{=} \PY{n}{dem}\PY{p}{(}\PY{n}{bin200}\PY{p}{)}
\PY{n+nb}{print}\PY{p}{(}\PY{l+s+s2}{\PYZdq{}}\PY{l+s+s2}{Media muestral: }\PY{l+s+si}{\PYZpc{}.2f}\PY{l+s+s2}{ | Desviacion estandar muestral: }\PY{l+s+si}{\PYZpc{}.2f}\PY{l+s+s2}{\PYZdq{}} \PY{o}{\PYZpc{}} \PY{p}{(}\PY{n}{bin200\PYZus{}mm}\PY{p}{,} \PY{n}{bin200\PYZus{}dem}\PY{p}{)}\PY{p}{)}
\PY{n}{histograma}\PY{p}{(}\PY{n}{bin200}\PY{p}{,} \PY{p}{[}\PY{l+m+mi}{1}\PY{p}{]}\PY{p}{)}
\end{Verbatim}
\end{tcolorbox}

    \begin{Verbatim}[commandchars=\\\{\}]
Media muestral: 79.19 | Desviacion estandar muestral: 6.82
    \end{Verbatim}

    \begin{center}
    \adjustimage{max size={0.9\linewidth}{0.9\paperheight}}{output_files/output_32_1.png}
    \end{center}
    { \hspace*{\fill} \\}
    
    La forma de la campana caracteristica de una muestra con distribucion
normal es mas marcada. Las 100 repeticiones del experimento hacen que
haya resultados que se puedan promediar con el fin de obtener mayor
precision en el mismo. Es asi que lo que se puede observar es que el
promedio de todas estas repeticiones con probabilidad p (en este caso se
tomo un p de valor 0.4) tienden o convergen al valor de la media (cuyo
valor teorico es de 80 para esta muestra en particular).

Luego, con la media y desvío estandard muestrales estandarizaremos cada
uno de los datos de la muestra y graficamos el histograma de la muestra
estandarizada obtenida:

    \begin{tcolorbox}[breakable, size=fbox, boxrule=1pt, pad at break*=1mm,colback=cellbackground, colframe=cellborder]
\prompt{In}{incolor}{8}{\boxspacing}
\begin{Verbatim}[commandchars=\\\{\}]
\PY{n}{bin200\PYZus{}normalizada} \PY{o}{=} \PY{p}{[} \PY{p}{(}\PY{n}{x} \PY{o}{\PYZhy{}} \PY{n}{bin200\PYZus{}mm}\PY{p}{)}\PY{o}{/}\PY{n}{bin200\PYZus{}dem} \PY{k}{for} \PY{n}{x} \PY{o+ow}{in} \PY{n}{bin200}\PY{p}{]}
\PY{n}{histograma}\PY{p}{(}\PY{n}{bin200\PYZus{}normalizada}\PY{p}{,} \PY{p}{[}\PY{l+m+mf}{0.3}\PY{p}{]}\PY{p}{)}
\end{Verbatim}
\end{tcolorbox}

    \begin{center}
    \adjustimage{max size={0.9\linewidth}{0.9\paperheight}}{output_files/output_34_0.png}
    \end{center}
    { \hspace*{\fill} \\}
    
    Por ultimo, computamos las medias muestrales para cada una de las
muestras anteriores y las escribimos en la siguiente tabla junto con su
error relativo con respecto a la media de la distribución binomial de
base:

    \begin{tcolorbox}[breakable, size=fbox, boxrule=1pt, pad at break*=1mm,colback=cellbackground, colframe=cellborder]
\prompt{In}{incolor}{9}{\boxspacing}
\begin{Verbatim}[commandchars=\\\{\}]
\PY{k+kn}{from} \PY{n+nn}{IPython}\PY{n+nn}{.}\PY{n+nn}{display} \PY{k+kn}{import} \PY{n}{display}\PY{p}{,} \PY{n}{Markdown}
\PY{n}{display}\PY{p}{(}\PY{n}{Markdown}\PY{p}{(}\PY{l+s+s1}{\PYZsq{}\PYZsq{}\PYZsq{}}
\PY{l+s+s1}{| Tamaño |  Media Muestral | Media Binomial | Err. Rel |}
\PY{l+s+s1}{|\PYZhy{}\PYZhy{}\PYZhy{}\PYZhy{}\PYZhy{}\PYZhy{}\PYZhy{}\PYZhy{}|\PYZhy{}\PYZhy{}\PYZhy{}\PYZhy{}\PYZhy{}\PYZhy{}\PYZhy{}\PYZhy{}\PYZhy{}\PYZhy{}\PYZhy{}\PYZhy{}\PYZhy{}\PYZhy{}\PYZhy{}\PYZhy{}\PYZhy{}|\PYZhy{}\PYZhy{}\PYZhy{}\PYZhy{}\PYZhy{}\PYZhy{}\PYZhy{}\PYZhy{}\PYZhy{}\PYZhy{}\PYZhy{}\PYZhy{}\PYZhy{}\PYZhy{}\PYZhy{}\PYZhy{}|\PYZhy{}\PYZhy{}\PYZhy{}\PYZhy{}\PYZhy{}\PYZhy{}\PYZhy{}\PYZhy{}\PYZhy{}\PYZhy{}|}
\PY{l+s+s1}{| }\PY{l+s+si}{\PYZpc{}d}\PY{l+s+s1}{     |  }\PY{l+s+si}{\PYZpc{}.2f}\PY{l+s+s1}{           | }\PY{l+s+si}{\PYZpc{}d}\PY{l+s+s1}{             | }\PY{l+s+si}{\PYZpc{}.3f}\PY{l+s+s1}{     |}
\PY{l+s+s1}{| }\PY{l+s+si}{\PYZpc{}d}\PY{l+s+s1}{     |  }\PY{l+s+si}{\PYZpc{}.2f}\PY{l+s+s1}{           | }\PY{l+s+si}{\PYZpc{}d}\PY{l+s+s1}{             | }\PY{l+s+si}{\PYZpc{}.3f}\PY{l+s+s1}{     |}
\PY{l+s+s1}{| }\PY{l+s+si}{\PYZpc{}d}\PY{l+s+s1}{     |  }\PY{l+s+si}{\PYZpc{}.2f}\PY{l+s+s1}{           | }\PY{l+s+si}{\PYZpc{}d}\PY{l+s+s1}{             | }\PY{l+s+si}{\PYZpc{}.3f}\PY{l+s+s1}{     |}
\PY{l+s+s1}{| }\PY{l+s+si}{\PYZpc{}d}\PY{l+s+s1}{     |  }\PY{l+s+si}{\PYZpc{}.2f}\PY{l+s+s1}{           | }\PY{l+s+si}{\PYZpc{}d}\PY{l+s+s1}{             | }\PY{l+s+si}{\PYZpc{}.3f}\PY{l+s+s1}{     |}
\PY{l+s+s1}{| }\PY{l+s+si}{\PYZpc{}d}\PY{l+s+s1}{     |  }\PY{l+s+si}{\PYZpc{}.2f}\PY{l+s+s1}{           | }\PY{l+s+si}{\PYZpc{}d}\PY{l+s+s1}{             | }\PY{l+s+si}{\PYZpc{}.3f}\PY{l+s+s1}{     |  }\PY{l+s+s1}{\PYZsq{}\PYZsq{}\PYZsq{}} \PY{o}{\PYZpc{}} \PY{p}{(} \PY{l+m+mi}{10}\PY{p}{,} \PY{n}{mm}\PY{p}{(}\PY{n}{bin10}\PY{p}{)}\PY{p}{,}   \PY{l+m+mi}{4}\PY{p}{,} \PY{n+nb}{abs}\PY{p}{(}\PY{n}{mm}\PY{p}{(}\PY{n}{bin10}\PY{p}{)}\PY{o}{/}\PY{l+m+mi}{4} \PY{o}{\PYZhy{}} \PY{l+m+mi}{1}\PY{p}{)}\PY{p}{,}
        \PY{l+m+mi}{20}\PY{p}{,} \PY{n}{mm}\PY{p}{(}\PY{n}{bin20}\PY{p}{)}\PY{p}{,}   \PY{l+m+mi}{8}\PY{p}{,} \PY{n+nb}{abs}\PY{p}{(}\PY{n}{mm}\PY{p}{(}\PY{n}{bin20}\PY{p}{)}\PY{o}{/}\PY{l+m+mi}{8} \PY{o}{\PYZhy{}} \PY{l+m+mi}{1}\PY{p}{)}\PY{p}{,}
        \PY{l+m+mi}{50}\PY{p}{,} \PY{n}{mm}\PY{p}{(}\PY{n}{bin50}\PY{p}{)}\PY{p}{,}  \PY{l+m+mi}{20}\PY{p}{,} \PY{n+nb}{abs}\PY{p}{(}\PY{n}{mm}\PY{p}{(}\PY{n}{bin50}\PY{p}{)}\PY{o}{/}\PY{l+m+mi}{20} \PY{o}{\PYZhy{}} \PY{l+m+mi}{1}\PY{p}{)}\PY{p}{,}
       \PY{l+m+mi}{100}\PY{p}{,} \PY{n}{mm}\PY{p}{(}\PY{n}{bin100}\PY{p}{)}\PY{p}{,} \PY{l+m+mi}{40}\PY{p}{,} \PY{n+nb}{abs}\PY{p}{(}\PY{n}{mm}\PY{p}{(}\PY{n}{bin100}\PY{p}{)}\PY{o}{/}\PY{l+m+mi}{40} \PY{o}{\PYZhy{}} \PY{l+m+mi}{1}\PY{p}{)}\PY{p}{,}
       \PY{l+m+mi}{200}\PY{p}{,} \PY{n}{mm}\PY{p}{(}\PY{n}{bin200}\PY{p}{)}\PY{p}{,} \PY{l+m+mi}{80}\PY{p}{,} \PY{n+nb}{abs}\PY{p}{(}\PY{n}{mm}\PY{p}{(}\PY{n}{bin200}\PY{p}{)}\PY{o}{/}\PY{l+m+mi}{80} \PY{o}{\PYZhy{}} \PY{l+m+mi}{1}\PY{p}{)}\PY{p}{)}\PY{p}{)}\PY{p}{)}
\end{Verbatim}
\end{tcolorbox}

    \begin{longtable}[]{@{}llll@{}}
\toprule
Tamaño & Media Muestral & Media Binomial & Err. Rel\tabularnewline
\midrule
\endhead
10 & 4.43 & 4 & 0.107\tabularnewline
20 & 7.90 & 8 & 0.012\tabularnewline
50 & 20.31 & 20 & 0.015\tabularnewline
100 & 40.14 & 40 & 0.004\tabularnewline
200 & 79.19 & 80 & 0.010\tabularnewline
\bottomrule
\end{longtable}

    
    Repitiendo la experiencia muchas veces podemos observar que, si bien no
de manera exacta dada la naturaleza probabilística de nuestro
experimento, al aumentar la cantidad de experimentos Bernoulli de cada
muestra y mantener el parámetros de probabilidad de éxito, el error
relativo tiende a disminuir.

    \hypertarget{parte-4-estaduxedstica-inferencial}{%
\section{Parte 4: Estadística
inferencial}\label{parte-4-estaduxedstica-inferencial}}

Tiene como objetivo hacer inferencias sobre muestras generadas, para asi
obtener informacion sobre sus correspondientes distribuciones. Se
importar las funciones auxiliares que hemos implementado para simular
las muestras de variables aleatorias:

    \begin{tcolorbox}[breakable, size=fbox, boxrule=1pt, pad at break*=1mm,colback=cellbackground, colframe=cellborder]
\prompt{In}{incolor}{136}{\boxspacing}
\begin{Verbatim}[commandchars=\\\{\}]
\PY{k+kn}{import} \PY{n+nn}{sys}\PY{p}{;} \PY{n}{sys}\PY{o}{.}\PY{n}{path}\PY{o}{.}\PY{n}{insert}\PY{p}{(}\PY{l+m+mi}{0}\PY{p}{,} \PY{l+s+s1}{\PYZsq{}}\PY{l+s+s1}{..}\PY{l+s+s1}{\PYZsq{}}\PY{p}{)}
\PY{k+kn}{from} \PY{n+nn}{simulador} \PY{k+kn}{import} \PY{o}{*}
\PY{k+kn}{from} \PY{n+nn}{math} \PY{k+kn}{import} \PY{o}{*}
\end{Verbatim}
\end{tcolorbox}

    A continuacion, se generaran dos muestras con distribucion normal con
media igual a 100 y varianza igual a 5, pero una con 10 repeticiones y
la otra con 30 repeticiones. La primera muestra con n = 10 esta
representada en el siguiente histograma:

    \begin{tcolorbox}[breakable, size=fbox, boxrule=1pt, pad at break*=1mm,colback=cellbackground, colframe=cellborder]
\prompt{In}{incolor}{137}{\boxspacing}
\begin{Verbatim}[commandchars=\\\{\}]
\PY{o}{\PYZpc{}}\PY{k}{matplotlib} inline
\PY{k+kn}{from} \PY{n+nn}{graficas} \PY{k+kn}{import} \PY{n}{histograma}
\PY{n}{nor10} \PY{o}{=} \PY{n}{normal}\PY{p}{(}\PY{l+m+mi}{10}\PY{p}{,} \PY{l+m+mi}{100}\PY{p}{,} \PY{n}{sqrt}\PY{p}{(}\PY{l+m+mi}{5}\PY{p}{)}\PY{p}{)}
\PY{n}{histograma}\PY{p}{(}\PY{n}{nor10}\PY{p}{,} \PY{p}{[}\PY{l+m+mi}{1}\PY{p}{]}\PY{p}{)}
\end{Verbatim}
\end{tcolorbox}

    \begin{center}
    \adjustimage{max size={0.9\linewidth}{0.9\paperheight}}{output_files/output_41_0.png}
    \end{center}
    { \hspace*{\fill} \\}
    
    La segunda muestra con n = 30 esta representada en el siguiente
histograma:

    \begin{tcolorbox}[breakable, size=fbox, boxrule=1pt, pad at break*=1mm,colback=cellbackground, colframe=cellborder]
\prompt{In}{incolor}{138}{\boxspacing}
\begin{Verbatim}[commandchars=\\\{\}]
\PY{n}{nor30} \PY{o}{=} \PY{n}{normal}\PY{p}{(}\PY{l+m+mi}{30}\PY{p}{,} \PY{l+m+mi}{100}\PY{p}{,} \PY{n}{sqrt}\PY{p}{(}\PY{l+m+mi}{5}\PY{p}{)}\PY{p}{)}
\PY{n}{histograma}\PY{p}{(}\PY{n}{nor30}\PY{p}{,} \PY{p}{[}\PY{l+m+mi}{1}\PY{p}{]}\PY{p}{)}
\end{Verbatim}
\end{tcolorbox}

    \begin{center}
    \adjustimage{max size={0.9\linewidth}{0.9\paperheight}}{output_files/output_43_0.png}
    \end{center}
    { \hspace*{\fill} \\}
    
    Para las muestra con n = 10 y n = 30 respectivamente, se obtienen las
estimaciones puntuales de sus medias y de sus varianzas.

Como se vio a lo largo del curso la media muestral (o \(\bar x\)) es un
estimador insesgado de la media por lo que es el estimador que
utilizaremos.

La varianza muestral(o \(s^2\)) es un estimador insesgado de la varianza
por lo que es el estimador que utilizaremos. Ya que el estimador debe
recibir como input los datos de las muestras, tales estimaciones estan
representadas en la siguiente tabla :

    \begin{tcolorbox}[breakable, size=fbox, boxrule=1pt, pad at break*=1mm,colback=cellbackground, colframe=cellborder]
\prompt{In}{incolor}{139}{\boxspacing}
\begin{Verbatim}[commandchars=\\\{\}]
\PY{k+kn}{from} \PY{n+nn}{IPython}\PY{n+nn}{.}\PY{n+nn}{display} \PY{k+kn}{import} \PY{n}{display}\PY{p}{,} \PY{n}{Markdown}
\PY{k+kn}{from} \PY{n+nn}{analisis} \PY{k+kn}{import} \PY{n}{media\PYZus{}muestral} \PY{k}{as} \PY{n}{mm}
\PY{k+kn}{from} \PY{n+nn}{analisis} \PY{k+kn}{import} \PY{n}{varianza\PYZus{}muestral} \PY{k}{as} \PY{n}{vm}
\PY{n}{display}\PY{p}{(}\PY{n}{Markdown}\PY{p}{(}\PY{l+s+s1}{\PYZsq{}\PYZsq{}\PYZsq{}}
\PY{l+s+s1}{| Tamaño    | Media estimada | Varianza estimada|}
\PY{l+s+s1}{|\PYZhy{}\PYZhy{}\PYZhy{}\PYZhy{}\PYZhy{}\PYZhy{}\PYZhy{}\PYZhy{}\PYZhy{}\PYZhy{}\PYZhy{}|\PYZhy{}\PYZhy{}\PYZhy{}\PYZhy{}\PYZhy{}\PYZhy{}\PYZhy{}\PYZhy{}\PYZhy{}\PYZhy{}\PYZhy{}\PYZhy{}\PYZhy{}\PYZhy{}\PYZhy{} |\PYZhy{}\PYZhy{}\PYZhy{}\PYZhy{}\PYZhy{}\PYZhy{}\PYZhy{}\PYZhy{}\PYZhy{}\PYZhy{}\PYZhy{}\PYZhy{}\PYZhy{}\PYZhy{}\PYZhy{}\PYZhy{}\PYZhy{}\PYZhy{}|}
\PY{l+s+s1}{| }\PY{l+s+si}{\PYZpc{}d}\PY{l+s+s1}{        | }\PY{l+s+si}{\PYZpc{}.2f}\PY{l+s+s1}{           | }\PY{l+s+si}{\PYZpc{}.2f}\PY{l+s+s1}{             |}
\PY{l+s+s1}{| }\PY{l+s+si}{\PYZpc{}d}\PY{l+s+s1}{        | }\PY{l+s+si}{\PYZpc{}.2f}\PY{l+s+s1}{           | }\PY{l+s+si}{\PYZpc{}.2f}\PY{l+s+s1}{             |}
\PY{l+s+s1}{\PYZsq{}\PYZsq{}\PYZsq{}} \PY{o}{\PYZpc{}} \PY{p}{(}\PY{l+m+mi}{10}\PY{p}{,} \PY{n}{mm}\PY{p}{(}\PY{n}{nor10}\PY{p}{)}\PY{p}{,} \PY{n}{vm}\PY{p}{(}\PY{n}{nor10}\PY{p}{)}\PY{p}{,}
       \PY{l+m+mi}{30}\PY{p}{,} \PY{n}{mm}\PY{p}{(}\PY{n}{nor30}\PY{p}{)}\PY{p}{,} \PY{n}{vm}\PY{p}{(}\PY{n}{nor30}\PY{p}{)}\PY{p}{)}\PY{p}{)}\PY{p}{)}
\end{Verbatim}
\end{tcolorbox}

    \begin{longtable}[]{@{}lll@{}}
\toprule
Tamaño & Media estimada & Varianza estimada\tabularnewline
\midrule
\endhead
10 & 99.36 & 5.41\tabularnewline
30 & 100.01 & 6.32\tabularnewline
\bottomrule
\end{longtable}

    
    Ahora, se pide que dado que se conoce el dato de que la distribucion
tiene varianza 5, se obtengan intervalos de confianza del 95\% y del
98\% para la media estimada de ambas muestras. Utilizaremos la librería
\texttt{scipy} para obtener los valores de la inversa de la función
distribución para una una distribución normal estandar, es decir Z
\textasciitilde{} N(0,1), correspondientes a un intervalo de confianza
\(1 - \alpha\), o lo que es lo mismo, el valor de \(Z_{\alpha/2}\).

Para la primera muestra, el intervalo de 95\% de confianza es:

    \begin{tcolorbox}[breakable, size=fbox, boxrule=1pt, pad at break*=1mm,colback=cellbackground, colframe=cellborder]
\prompt{In}{incolor}{140}{\boxspacing}
\begin{Verbatim}[commandchars=\\\{\}]
\PY{k+kn}{from} \PY{n+nn}{scipy}\PY{n+nn}{.}\PY{n+nn}{stats} \PY{k+kn}{import} \PY{n}{norm}
\PY{n}{marg} \PY{o}{=} \PY{k}{lambda} \PY{n}{a}\PY{p}{,} \PY{n}{n} \PY{p}{:} \PY{n}{norm}\PY{o}{.}\PY{n}{ppf}\PY{p}{(}\PY{p}{(}\PY{l+m+mi}{1} \PY{o}{\PYZhy{}} \PY{n}{a}\PY{p}{)}\PY{o}{/}\PY{l+m+mi}{2}\PY{p}{)} \PY{o}{*} \PY{n}{sqrt}\PY{p}{(}\PY{l+m+mi}{5}\PY{p}{)} \PY{o}{/} \PY{n}{sqrt}\PY{p}{(}\PY{n}{n}\PY{p}{)}
\PY{n}{interv} \PY{o}{=} \PY{k}{lambda} \PY{n}{s}\PY{p}{,} \PY{n}{a} \PY{p}{:} \PY{p}{(}\PY{n}{mm}\PY{p}{(}\PY{n}{s}\PY{p}{)} \PY{o}{+} \PY{n}{marg}\PY{p}{(}\PY{n}{a}\PY{p}{,} \PY{n+nb}{len}\PY{p}{(}\PY{n}{s}\PY{p}{)}\PY{p}{)}\PY{p}{,} \PY{n}{mm}\PY{p}{(}\PY{n}{s}\PY{p}{)} \PY{o}{\PYZhy{}} \PY{n}{marg}\PY{p}{(}\PY{n}{a}\PY{p}{,} \PY{n+nb}{len}\PY{p}{(}\PY{n}{s}\PY{p}{)}\PY{p}{)}\PY{p}{)}
\PY{n}{interv}\PY{p}{(}\PY{n}{nor10}\PY{p}{,} \PY{o}{.}\PY{l+m+mi}{95}\PY{p}{)}
\end{Verbatim}
\end{tcolorbox}

            \begin{tcolorbox}[breakable, size=fbox, boxrule=.5pt, pad at break*=1mm, opacityfill=0]
\prompt{Out}{outcolor}{140}{\boxspacing}
\begin{Verbatim}[commandchars=\\\{\}]
(97.9711346260824, 100.74294227478177)
\end{Verbatim}
\end{tcolorbox}
        
    Para la misma muestra, obtenemos el intervalo de 98\% de confianza para
la media:

    \begin{tcolorbox}[breakable, size=fbox, boxrule=1pt, pad at break*=1mm,colback=cellbackground, colframe=cellborder]
\prompt{In}{incolor}{141}{\boxspacing}
\begin{Verbatim}[commandchars=\\\{\}]
\PY{n}{interv}\PY{p}{(}\PY{n}{nor10}\PY{p}{,} \PY{o}{.}\PY{l+m+mi}{98}\PY{p}{)}
\end{Verbatim}
\end{tcolorbox}

            \begin{tcolorbox}[breakable, size=fbox, boxrule=.5pt, pad at break*=1mm, opacityfill=0]
\prompt{Out}{outcolor}{141}{\boxspacing}
\begin{Verbatim}[commandchars=\\\{\}]
(97.7120620932989, 101.00201480756527)
\end{Verbatim}
\end{tcolorbox}
        
    De la misma forma obtenemos los intervalos de confianza para la muestra
de 30 datos con 95\% de confianza:

    \begin{tcolorbox}[breakable, size=fbox, boxrule=1pt, pad at break*=1mm,colback=cellbackground, colframe=cellborder]
\prompt{In}{incolor}{142}{\boxspacing}
\begin{Verbatim}[commandchars=\\\{\}]
\PY{n}{interv}\PY{p}{(}\PY{n}{nor30}\PY{p}{,} \PY{o}{.}\PY{l+m+mi}{95}\PY{p}{)}
\end{Verbatim}
\end{tcolorbox}

            \begin{tcolorbox}[breakable, size=fbox, boxrule=.5pt, pad at break*=1mm, opacityfill=0]
\prompt{Out}{outcolor}{142}{\boxspacing}
\begin{Verbatim}[commandchars=\\\{\}]
(99.20858170831391, 100.80888560043235)
\end{Verbatim}
\end{tcolorbox}
        
    Y finalmente un intervalo del 98\% de confianza para la muestra de 30
datos:

    \begin{tcolorbox}[breakable, size=fbox, boxrule=1pt, pad at break*=1mm,colback=cellbackground, colframe=cellborder]
\prompt{In}{incolor}{143}{\boxspacing}
\begin{Verbatim}[commandchars=\\\{\}]
\PY{n}{interv}\PY{p}{(}\PY{n}{nor30}\PY{p}{,} \PY{o}{.}\PY{l+m+mi}{98}\PY{p}{)}
\end{Verbatim}
\end{tcolorbox}

            \begin{tcolorbox}[breakable, size=fbox, boxrule=.5pt, pad at break*=1mm, opacityfill=0]
\prompt{Out}{outcolor}{143}{\boxspacing}
\begin{Verbatim}[commandchars=\\\{\}]
(99.05900611177172, 100.95846119697454)
\end{Verbatim}
\end{tcolorbox}
        
    Como es posible observar ante la repetición de la experiencia, la gran
mayoría de las veces, pero no todas, las medias de las distribuciones de
base se encuentran en el intervalo de confianza inferido a partir de las
muestras y la varianza conocida.

Ahora repetiremos la experienza pero utilizando la varianza muestral
como estimador. Por ello redefiniremos nuestro método utilizando la
distribución T de Student y calcularemos el intervalo de 95\% de
confianza para la muestra de diez datos de población normal con varianza
desconocida:

    \begin{tcolorbox}[breakable, size=fbox, boxrule=1pt, pad at break*=1mm,colback=cellbackground, colframe=cellborder]
\prompt{In}{incolor}{144}{\boxspacing}
\begin{Verbatim}[commandchars=\\\{\}]
\PY{k+kn}{from} \PY{n+nn}{scipy}\PY{n+nn}{.}\PY{n+nn}{stats} \PY{k+kn}{import} \PY{n}{t}
\PY{n}{marg2} \PY{o}{=} \PY{k}{lambda} \PY{n}{a}\PY{p}{,} \PY{n}{s} \PY{p}{:} \PY{n}{t}\PY{o}{.}\PY{n}{ppf}\PY{p}{(}\PY{p}{(}\PY{l+m+mi}{1} \PY{o}{\PYZhy{}} \PY{n}{a}\PY{p}{)}\PY{o}{/}\PY{l+m+mi}{2}\PY{p}{,} \PY{n+nb}{len}\PY{p}{(}\PY{n}{s}\PY{p}{)} \PY{o}{\PYZhy{}} \PY{l+m+mi}{1}\PY{p}{)} \PY{o}{*} \PY{n}{sqrt}\PY{p}{(}\PY{n}{vm}\PY{p}{(}\PY{n}{s}\PY{p}{)}\PY{o}{/}\PY{n+nb}{len}\PY{p}{(}\PY{n}{s}\PY{p}{)}\PY{p}{)}
\PY{n}{interv2} \PY{o}{=} \PY{k}{lambda} \PY{n}{s}\PY{p}{,} \PY{n}{a} \PY{p}{:} \PY{p}{(}\PY{n}{mm}\PY{p}{(}\PY{n}{s}\PY{p}{)} \PY{o}{+} \PY{n}{marg2}\PY{p}{(}\PY{n}{a}\PY{p}{,} \PY{n}{s}\PY{p}{)}\PY{p}{,} \PY{n}{mm}\PY{p}{(}\PY{n}{s}\PY{p}{)} \PY{o}{\PYZhy{}} \PY{n}{marg2}\PY{p}{(}\PY{n}{a}\PY{p}{,} \PY{n}{s}\PY{p}{)}\PY{p}{)}
\PY{n}{interv2}\PY{p}{(}\PY{n}{nor10}\PY{p}{,} \PY{o}{.}\PY{l+m+mi}{95}\PY{p}{)}
\end{Verbatim}
\end{tcolorbox}

            \begin{tcolorbox}[breakable, size=fbox, boxrule=.5pt, pad at break*=1mm, opacityfill=0]
\prompt{Out}{outcolor}{144}{\boxspacing}
\begin{Verbatim}[commandchars=\\\{\}]
(97.69323182468034, 101.02084507618383)
\end{Verbatim}
\end{tcolorbox}
        
    El intervalo de 98\% de confianza para la misma muestra será:

    \begin{tcolorbox}[breakable, size=fbox, boxrule=1pt, pad at break*=1mm,colback=cellbackground, colframe=cellborder]
\prompt{In}{incolor}{145}{\boxspacing}
\begin{Verbatim}[commandchars=\\\{\}]
\PY{n}{interv2}\PY{p}{(}\PY{n}{nor10}\PY{p}{,} \PY{o}{.}\PY{l+m+mi}{98}\PY{p}{)}
\end{Verbatim}
\end{tcolorbox}

            \begin{tcolorbox}[breakable, size=fbox, boxrule=.5pt, pad at break*=1mm, opacityfill=0]
\prompt{Out}{outcolor}{145}{\boxspacing}
\begin{Verbatim}[commandchars=\\\{\}]
(97.28188328957371, 101.43219361129046)
\end{Verbatim}
\end{tcolorbox}
        
    Finalmente calcularemos los intervalos de confianza de la media,
asumiendo una distribución normal con varianza desconocida, que se
infieren a partir de la muestra de 30 datos. Primero con un 95\% de
confianza:

    \begin{tcolorbox}[breakable, size=fbox, boxrule=1pt, pad at break*=1mm,colback=cellbackground, colframe=cellborder]
\prompt{In}{incolor}{146}{\boxspacing}
\begin{Verbatim}[commandchars=\\\{\}]
\PY{n}{interv2}\PY{p}{(}\PY{n}{nor30}\PY{p}{,} \PY{o}{.}\PY{l+m+mi}{95}\PY{p}{)}
\end{Verbatim}
\end{tcolorbox}

            \begin{tcolorbox}[breakable, size=fbox, boxrule=.5pt, pad at break*=1mm, opacityfill=0]
\prompt{Out}{outcolor}{146}{\boxspacing}
\begin{Verbatim}[commandchars=\\\{\}]
(99.06963427474926, 100.947833033997)
\end{Verbatim}
\end{tcolorbox}
        
    Y con 98\% de confianza:

    \begin{tcolorbox}[breakable, size=fbox, boxrule=1pt, pad at break*=1mm,colback=cellbackground, colframe=cellborder]
\prompt{In}{incolor}{147}{\boxspacing}
\begin{Verbatim}[commandchars=\\\{\}]
\PY{n}{interv2}\PY{p}{(}\PY{n}{nor30}\PY{p}{,} \PY{o}{.}\PY{l+m+mi}{98}\PY{p}{)}
\end{Verbatim}
\end{tcolorbox}

            \begin{tcolorbox}[breakable, size=fbox, boxrule=.5pt, pad at break*=1mm, opacityfill=0]
\prompt{Out}{outcolor}{147}{\boxspacing}
\begin{Verbatim}[commandchars=\\\{\}]
(98.87825779762437, 101.13920951112189)
\end{Verbatim}
\end{tcolorbox}
        
    Así hemos verificado empíricamente que repitiendo la experiencia, los
intervalos bajo la asunción de de una varianza conocida son menores
(para el mismo porcentaje de confianza) que los obtenidos
desconociéndola y reemplazándola por la varianza muestral.

A continuación probaremos si nuestras muestras nos permiten inferir que
provienen de distribuciones normales cuya varianza sea mayor a algún
valor dado. Para ello definimos una función auxiliar que calcula el
límite inferior de un intervalo de confianza unilateral con la ayuda de
la distribución Chi cuadrado.

    \begin{tcolorbox}[breakable, size=fbox, boxrule=1pt, pad at break*=1mm,colback=cellbackground, colframe=cellborder]
\prompt{In}{incolor}{148}{\boxspacing}
\begin{Verbatim}[commandchars=\\\{\}]
\PY{k+kn}{from} \PY{n+nn}{scipy}\PY{n+nn}{.}\PY{n+nn}{stats} \PY{k+kn}{import} \PY{n}{chi2}
\PY{n}{lowbound} \PY{o}{=} \PY{k}{lambda} \PY{n}{s}\PY{p}{,} \PY{n}{p}\PY{p}{:} \PY{p}{(} \PY{n+nb}{len}\PY{p}{(}\PY{n}{s}\PY{p}{)} \PY{o}{\PYZhy{}} \PY{l+m+mi}{1} \PY{p}{)} \PY{o}{*} \PY{n}{vm}\PY{p}{(}\PY{n}{s}\PY{p}{)} \PY{o}{/} \PY{n}{chi2}\PY{o}{.}\PY{n}{ppf}\PY{p}{(}\PY{n}{p}\PY{p}{,} \PY{n+nb}{len}\PY{p}{(}\PY{n}{s}\PY{p}{)} \PY{o}{\PYZhy{}} \PY{l+m+mi}{1}\PY{p}{)}
\end{Verbatim}
\end{tcolorbox}

    Para probar a continuación con un 99\% de confianza que la muestra de 30
datos proviene de una distribución normal cuya varianza es mayor a 5, su
límite inferior unilateral debe ser entonces también mayor.

    \begin{tcolorbox}[breakable, size=fbox, boxrule=1pt, pad at break*=1mm,colback=cellbackground, colframe=cellborder]
\prompt{In}{incolor}{149}{\boxspacing}
\begin{Verbatim}[commandchars=\\\{\}]
\PY{n+nb}{print}\PY{p}{(}\PY{n}{lowbound}\PY{p}{(}\PY{n}{nor30}\PY{p}{,} \PY{o}{.}\PY{l+m+mi}{99}\PY{p}{)}\PY{p}{)}
\end{Verbatim}
\end{tcolorbox}

    \begin{Verbatim}[commandchars=\\\{\}]
3.6989854435135374
    \end{Verbatim}

    Finalmente terminaremos nuestro trabajo realizando una prueba de bondad
de ajuste, para la cual también nuestro estimador se basará en la
distribución Chi cuadrado.

Generaremos primero los límites de las clases para un anchos de banda de
0.5

    \begin{tcolorbox}[breakable, size=fbox, boxrule=1pt, pad at break*=1mm,colback=cellbackground, colframe=cellborder]
\prompt{In}{incolor}{150}{\boxspacing}
\begin{Verbatim}[commandchars=\\\{\}]
\PY{k+kn}{from} \PY{n+nn}{numpy} \PY{k+kn}{import} \PY{n}{arange}
\PY{n}{gaps} \PY{o}{=} \PY{n}{arange}\PY{p}{(}\PY{n+nb}{min}\PY{p}{(}\PY{n}{nor30}\PY{p}{)}\PY{p}{,} \PY{n+nb}{max}\PY{p}{(}\PY{n}{nor30}\PY{p}{)}\PY{p}{,} \PY{l+m+mf}{0.5}\PY{p}{)}
\PY{n+nb}{print}\PY{p}{(}\PY{n}{gaps}\PY{p}{)}
\end{Verbatim}
\end{tcolorbox}

    \begin{Verbatim}[commandchars=\\\{\}]
[ 93.42722231  93.92722231  94.42722231  94.92722231  95.42722231
  95.92722231  96.42722231  96.92722231  97.42722231  97.92722231
  98.42722231  98.92722231  99.42722231  99.92722231 100.42722231
 100.92722231 101.42722231 101.92722231 102.42722231 102.92722231
 103.42722231 103.92722231 104.42722231 104.92722231 105.42722231]
    \end{Verbatim}

    Obtendremos ahora las frecuencias observadas para cada clase:

    \begin{tcolorbox}[breakable, size=fbox, boxrule=1pt, pad at break*=1mm,colback=cellbackground, colframe=cellborder]
\prompt{In}{incolor}{151}{\boxspacing}
\begin{Verbatim}[commandchars=\\\{\}]
\PY{n}{obs} \PY{o}{=} \PY{p}{[}\PY{n+nb}{len}\PY{p}{(}\PY{p}{[}\PY{n}{x} \PY{k}{for} \PY{n}{x} \PY{o+ow}{in} \PY{n}{nor30} \PY{k}{if} \PY{n}{x} \PY{o}{\PYZgt{}}\PY{o}{=} \PY{n}{l} \PY{o+ow}{and} \PY{n}{x} \PY{o}{\PYZlt{}} \PY{n}{l} \PY{o}{+} \PY{l+m+mf}{0.5}\PY{p}{]}\PY{p}{)} \PY{k}{for} \PY{n}{l} \PY{o+ow}{in} \PY{n}{gaps}\PY{p}{]}
\PY{n+nb}{print}\PY{p}{(}\PY{n}{obs}\PY{p}{)}
\end{Verbatim}
\end{tcolorbox}

    \begin{Verbatim}[commandchars=\\\{\}]
[1, 0, 0, 0, 0, 0, 1, 2, 1, 2, 1, 5, 3, 0, 4, 3, 2, 1, 1, 0, 1, 0, 1, 0, 1]
    \end{Verbatim}

    Calcularemos entonces las frecuencias esperadas teniendo en cuenta que
la muestra proviene de una distribución normal:

    \begin{tcolorbox}[breakable, size=fbox, boxrule=1pt, pad at break*=1mm,colback=cellbackground, colframe=cellborder]
\prompt{In}{incolor}{152}{\boxspacing}
\begin{Verbatim}[commandchars=\\\{\}]
\PY{n}{esp} \PY{o}{=} \PY{p}{[}\PY{n+nb}{round}\PY{p}{(}\PY{p}{(}\PY{n}{norm}\PY{o}{.}\PY{n}{cdf}\PY{p}{(}\PY{n}{x} \PY{o}{+} \PY{l+m+mf}{0.5}\PY{p}{,} \PY{n}{loc}\PY{o}{=}\PY{l+m+mi}{100}\PY{p}{,} \PY{n}{scale}\PY{o}{=}\PY{n}{sqrt}\PY{p}{(}\PY{l+m+mi}{5}\PY{p}{)}\PY{p}{)} \PYZbs{}
              \PY{o}{\PYZhy{}} \PY{n}{norm}\PY{o}{.}\PY{n}{cdf}\PY{p}{(}\PY{n}{x}\PY{p}{,} \PY{n}{loc}\PY{o}{=}\PY{l+m+mi}{100}\PY{p}{,} \PY{n}{scale}\PY{o}{=}\PY{n}{sqrt}\PY{p}{(}\PY{l+m+mi}{5}\PY{p}{)}\PY{p}{)}\PY{p}{)} \PYZbs{}
                  \PY{o}{*} \PY{l+m+mi}{30}\PY{p}{)} \PY{k}{for} \PY{n}{x} \PY{o+ow}{in} \PY{n}{gaps}\PY{p}{]}
\PY{n+nb}{print}\PY{p}{(}\PY{n}{esp}\PY{p}{)}
\end{Verbatim}
\end{tcolorbox}

    \begin{Verbatim}[commandchars=\\\{\}]
[0, 0, 0, 0, 0, 1, 1, 1, 2, 2, 2, 2, 3, 3, 3, 2, 2, 2, 1, 1, 1, 0, 0, 0, 0]
    \end{Verbatim}

    Para que la prueba de bondad de ajuste funcione, es condición que las
clases tengan frecuencias mayores o iguales a 5, por lo que realizaremos
una unificación de clases hasta que ambas series, observadas y
esperadas, cumplan con lo requerido.

    \begin{tcolorbox}[breakable, size=fbox, boxrule=1pt, pad at break*=1mm,colback=cellbackground, colframe=cellborder]
\prompt{In}{incolor}{153}{\boxspacing}
\begin{Verbatim}[commandchars=\\\{\}]
\PY{n}{obs2} \PY{o}{=} \PY{p}{[}\PY{p}{]}
\PY{n}{esp2} \PY{o}{=} \PY{p}{[}\PY{p}{]}
\PY{n}{obsacm} \PY{o}{=} \PY{l+m+mi}{0}
\PY{n}{espacm} \PY{o}{=} \PY{l+m+mi}{0}
\PY{k}{for} \PY{n}{pos} \PY{o+ow}{in} \PY{n+nb}{range}\PY{p}{(}\PY{n+nb}{len}\PY{p}{(}\PY{n}{obs}\PY{p}{)}\PY{p}{)}\PY{p}{:}
    \PY{n}{obsacm} \PY{o}{+}\PY{o}{=} \PY{n}{obs}\PY{p}{[}\PY{n}{pos}\PY{p}{]}
    \PY{n}{espacm} \PY{o}{+}\PY{o}{=} \PY{n}{esp}\PY{p}{[}\PY{n}{pos}\PY{p}{]}
    \PY{k}{if} \PY{n}{obsacm} \PY{o}{\PYZgt{}}\PY{o}{=} \PY{l+m+mi}{5} \PY{o+ow}{and} \PY{n}{espacm} \PY{o}{\PYZgt{}}\PY{o}{=} \PY{l+m+mi}{5}\PY{p}{:}
        \PY{n}{obs2}\PY{o}{.}\PY{n}{append}\PY{p}{(}\PY{n}{obsacm}\PY{p}{)}
        \PY{n}{esp2}\PY{o}{.}\PY{n}{append}\PY{p}{(}\PY{n}{espacm}\PY{p}{)}
        \PY{n}{obsacm} \PY{o}{=} \PY{l+m+mi}{0}
        \PY{n}{espacm} \PY{o}{=} \PY{l+m+mi}{0}
\PY{k}{if} \PY{n}{obsacm} \PY{o+ow}{or} \PY{n}{espacm}\PY{p}{:}
    \PY{n}{esp2}\PY{p}{[}\PY{o}{\PYZhy{}}\PY{l+m+mi}{1}\PY{p}{]} \PY{o}{+}\PY{o}{=} \PY{n}{espacm}
    \PY{n}{obs2}\PY{p}{[}\PY{o}{\PYZhy{}}\PY{l+m+mi}{1}\PY{p}{]} \PY{o}{+}\PY{o}{=} \PY{n}{obsacm}
\PY{n+nb}{print}\PY{p}{(}\PY{n}{obs2}\PY{p}{)}
\PY{n+nb}{print}\PY{p}{(}\PY{n}{esp2}\PY{p}{)}
\end{Verbatim}
\end{tcolorbox}

    \begin{Verbatim}[commandchars=\\\{\}]
[5, 8, 7, 10]
[5, 6, 9, 9]
    \end{Verbatim}

    Ahora si podemos calcular nuestro estimador como la sumatoria de los
errores cuadrados relativos:

    \begin{tcolorbox}[breakable, size=fbox, boxrule=1pt, pad at break*=1mm,colback=cellbackground, colframe=cellborder]
\prompt{In}{incolor}{154}{\boxspacing}
\begin{Verbatim}[commandchars=\\\{\}]
\PY{n}{squarerr} \PY{o}{=} \PY{n+nb}{sum}\PY{p}{(}\PY{p}{[}\PY{p}{(}\PY{n}{obs2}\PY{p}{[}\PY{n}{pos}\PY{p}{]} \PY{o}{\PYZhy{}} \PY{n}{val}\PY{p}{)}\PY{o}{*}\PY{o}{*}\PY{l+m+mi}{2}\PY{o}{/}\PY{n}{val} \PY{k}{for} \PY{n}{pos}\PY{p}{,} \PY{n}{val} \PY{o+ow}{in} \PY{n+nb}{enumerate}\PY{p}{(}\PY{n}{esp2}\PY{p}{)}\PY{p}{]}\PY{p}{)}
\PY{n+nb}{print}\PY{p}{(}\PY{n}{squarerr}\PY{p}{)}
\end{Verbatim}
\end{tcolorbox}

    \begin{Verbatim}[commandchars=\\\{\}]
1.2222222222222223
    \end{Verbatim}

    Finalmente, para rechazar con 99 por ciento de confianza la hipótesis de
que nuestra muestra aleatoria proviene de una población con una
distribución normal cuyos parámetros son los especificados, nuestro
estimador debiera ser mayor al Chi cuadrado que corresponde a un alfa de
0.01 y los grados de libertad que se deducen de la unificación de las
clases.

    \begin{tcolorbox}[breakable, size=fbox, boxrule=1pt, pad at break*=1mm,colback=cellbackground, colframe=cellborder]
\prompt{In}{incolor}{155}{\boxspacing}
\begin{Verbatim}[commandchars=\\\{\}]
\PY{n}{squarerr} \PY{o}{\PYZgt{}} \PY{n}{chi2}\PY{o}{.}\PY{n}{ppf}\PY{p}{(}\PY{o}{.}\PY{l+m+mi}{01}\PY{p}{,} \PY{n+nb}{len}\PY{p}{(}\PY{n}{esp2}\PY{p}{)} \PY{o}{\PYZhy{}} \PY{l+m+mi}{1}\PY{p}{)}
\end{Verbatim}
\end{tcolorbox}

            \begin{tcolorbox}[breakable, size=fbox, boxrule=.5pt, pad at break*=1mm, opacityfill=0]
\prompt{Out}{outcolor}{155}{\boxspacing}
\begin{Verbatim}[commandchars=\\\{\}]
True
\end{Verbatim}
\end{tcolorbox}
        

\section{Conclusiones}

La realizació de la presente práctica ha significado para nosotros 
una interesante aproximación a problemas relativos a la simulación
que son parte integral de un serio proyecto ingenieril. A la vez,
los conceptos de inferencia estadística han sido lo sufiente
introductorios para abordar la comprensión de ámbitos que los 
incluyen y que son crecientemente requisitos de la industria, 
como es el \textit{machine learning} y \textit{pattern recognition}.

No queremos desaprovechar la circunstancia para hacer notar el
agradecimiento al acompañamiento pedagógico en vista de la 
situación epidemiológica, que junto con nuestro esfuerzo ha 
permitido la continuidad de proyecto universitario.

\noindent\rule{\textwidth}{1pt}

\begin{thebibliography}{9}
\bibitem{bib:dek}
Dekking, F. \& otros. (2005). 
\textit{``A Modern Introduction to Probability and Statistics. ''}
London: Springer.

\bibitem{bib:test}
\textit{``Unit testing framework — Python 3.8.2 documentation''}.
 Recuperado de: https://docs.python.org/3/library/unittest.html

 \bibitem{bib:jupyter}
 \textit{``IPython Kernel for Jupyter''}.
 (Version 5.3.4; IPython Development Team).
 Recuperado de https://ipython.org

 \bibitem{bib:matplot}
 \textit{``Python plotting package''}.
 (Version 3.3.3; John D. Hunter, Michael Droettboom).
 Recuperado de https://matplotlib.org

 \bibitem{bib:scipylib}
 \textit{``SciPy: Scientific Library for Python''}.
 (Version 1.5.4).
 Recuperado de https://www.scipy.org

 \bibitem{bib:sphinx}
 \textit{``Python documentation generator''}.
 (Version 3.3.1; Georg Brandl).
 Recuperado de http://sphinx-doc.org/

 \bibitem{bib:merge}
 \textit{``A tool to merge / concatenate Jupyter (IPython) notebooks''}.
 (Version 0.0.4; John Bjorn Nelson).
 Recuperado de https://github.com/jbn/nbmerge

 \bibitem{bib:git}
 \textit{``GitHub: Where the World builds software''}.
  Recuperado de https://github.com/

 \bibitem{bib:doc}
 Goodger, D., van Rossum, G. (2001).
 \textit{``PEP 257 -- Docstring Conventions''}
 Recuperado de https://www.python.org/dev/peps/pep-0257/

\end{thebibliography}
    
\end{document}